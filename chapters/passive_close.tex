\section{被动关闭}
\label{sec:passive_close}

	\subsection{基本流程}
		在正常的被动关闭开始时,TCP控制块目前处于ESTABLISHED状态,此时接收到的TCP段都由\mintinline{C}{tcp_rcv_established}函数来处理,因此FIN段必定要做首部预测,当然预测一定不会通过,所以FIN段是走\textbf{慢速路径}处理的。

		在慢速路径中,首先,首先进行TCP选项的处理(假设存在),然后根据段的序号检测该FIN段是不是期望接收的段。如果是,则调用\mintinline{C}{tcp_fin}函数进行处理。如果不是,则说明在TCP段传输过程中出现了失序,因此将该FIN段缓存到乱序队列中,等待它之前的所有TCP段都到其后才能作处理。

	\subsection{\mintinline{C}{ESTABLISHED-->CLOSE_WAIT}}

		在这个过程中,服务器段主要接收到了客户端的FIN段,并且跳转到了\mintinline{C}{CLOSE_WAIT}状态。
		\subsubsection{函数调用关系}
		

		\subsubsection{\mintinline{C}{tcp_fin}}

			此时,首先调用的传输层的函数为\mintinline{C}{tcp_rcv_established},然后走慢速路径由\mintinline{C}{tcp_data_queue}函数处理。在处理的过程中,如果FIN段时预期接收的段,则调用\mintinline{C}{tcp_fin}函数处理,否则,将该段暂存到乱序队列中,等待它之前的TCP段到齐之后再做处理,这里我们不说函数\mintinline{C}{tcp_rcv_established}与\mintinline{C}{tcp_data_queue}了,这些主要是在数据传送阶段介绍的函数。我们直接介绍\mintinline{C}{tcp_fin}函数。

\begin{minted}[linenos]{C}
/*
Location:

	net/ipv4/tcp_input.c

Function:

  	Process the FIN bit. This now behaves as it is supposed to work
 	and the FIN takes effect when it is validly part of sequence
 	space. Not before when we get holes.
 
 	If we are ESTABLISHED, a received fin moves us to CLOSE-WAIT
 	(and thence onto LAST-ACK and finally, CLOSE, we never enter
 	TIME-WAIT)
 
 	If we are in FINWAIT-1, a received FIN indicates simultaneous
	close and we go into CLOSING (and later onto TIME-WAIT)
 
 	If we are in FINWAIT-2, a received FIN moves us to TIME-WAIT.

Parameters:

	sk:
*/
static void tcp_fin(struct sock *sk)
{
	struct tcp_sock *tp = tcp_sk(sk);

	inet_csk_schedule_ack(sk);

	sk->sk_shutdown |= RCV_SHUTDOWN;
	sock_set_flag(sk, SOCK_DONE);

	switch (sk->sk_state) {
		case TCP_SYN_RECV:
		case TCP_ESTABLISHED:
			/* Move to CLOSE_WAIT */
			tcp_set_state(sk, TCP_CLOSE_WAIT);
			inet_csk(sk)->icsk_ack.pingpong = 1;
			break;

		case TCP_CLOSE_WAIT:
		case TCP_CLOSING:
			/* Received a retransmission of the FIN, do
			 * nothing.
			 */
			break;
		case TCP_LAST_ACK:
			/* RFC793: Remain in the LAST-ACK state. */
			break;

		case TCP_FIN_WAIT1:
			/* This case occurs when a simultaneous close
			 * happens, we must ack the received FIN and
			 * enter the CLOSING state.
			 */
			tcp_send_ack(sk);
			tcp_set_state(sk, TCP_CLOSING);
			break;
		case TCP_FIN_WAIT2:
			/* Received a FIN -- send ACK and enter TIME_WAIT. */
			tcp_send_ack(sk);
			tcp_time_wait(sk, TCP_TIME_WAIT, 0);
			break;
		default:
			/* Only TCP_LISTEN and TCP_CLOSE are left, in these
			 * cases we should never reach this piece of code.
			 */
			pr_err("%s: Impossible, sk->sk_state=%d\n",
				   __func__, sk->sk_state);
			break;
	}

	/* It _is_ possible, that we have something out-of-order _after_ FIN.
	 * Probably, we should reset in this case. For now drop them.
	 */
	__skb_queue_purge(&tp->out_of_order_queue);
	if (tcp_is_sack(tp))
		tcp_sack_reset(&tp->rx_opt);
	sk_mem_reclaim(sk);

	if (!sock_flag(sk, SOCK_DEAD)) {
		sk->sk_state_change(sk);

		/* Do not send POLL_HUP for half duplex close. */
		if (sk->sk_shutdown == SHUTDOWN_MASK ||
		    sk->sk_state == TCP_CLOSE)
			sk_wake_async(sk, SOCK_WAKE_WAITD, POLL_HUP);
		else
			sk_wake_async(sk, SOCK_WAKE_WAITD, POLL_IN);
	}
}

\end{minted}
