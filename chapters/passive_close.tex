\section{被动关闭}
\label{sec:passive_close}

	\subsection{基本流程}
		在正常的被动关闭开始时,TCP控制块目前处于ESTABLISHED状态,此时接收到的TCP段都由\mintinline{C}{tcp_rcv_established}函数来处理,因此FIN段必定要做首部预测,当然预测一定不会通过,所以FIN段是走\textbf{慢速路径}处理的。

		在慢速路径中,首先,首先进行TCP选项的处理(假设存在),然后根据段的序号检测该FIN段是不是期望接收的段。如果是,则调用\mintinline{C}{tcp_fin}函数进行处理。如果不是,则说明在TCP段传输过程中出现了失序,因此将该FIN段缓存到乱序队列中,等待它之前的所有TCP段都到其后才能作处理。

	\subsection{\mintinline{C}{第一次握手:接收FIN}}

		在这个过程中,服务器段主要接收到了客户端的FIN段,并且跳转到了\mintinline{C}{CLOSE_WAIT}状态。
		\subsubsection{函数调用关系}
		

		\subsubsection{\mintinline{C}{tcp_fin}}

			此时,首先调用的传输层的函数为\mintinline{C}{tcp_rcv_established},然后走慢速路径由\mintinline{C}{tcp_data_queue}函数处理。在处理的过程中,如果FIN段时预期接收的段,则调用\mintinline{C}{tcp_fin}函数处理,否则,将该段暂存到乱序队列中,等待它之前的TCP段到齐之后再做处理,这里我们不说函数\mintinline{C}{tcp_rcv_established}与\mintinline{C}{tcp_data_queue}了,这些主要是在数据传送阶段介绍的函数。我们直接介绍\mintinline{C}{tcp_fin}函数。

\begin{minted}[linenos]{C}
/*
Location:

	net/ipv4/tcp_input.c

Function:

  	Process the FIN bit. This now behaves as it is supposed to work
 	and the FIN takes effect when it is validly part of sequence
 	space. Not before when we get holes.
 
 	If we are ESTABLISHED, a received fin moves us to CLOSE-WAIT
 	(and thence onto LAST-ACK and finally, CLOSE, we never enter
 	TIME-WAIT)
 
 	If we are in FINWAIT-1, a received FIN indicates simultaneous
	close and we go into CLOSING (and later onto TIME-WAIT)
 
 	If we are in FINWAIT-2, a received FIN moves us to TIME-WAIT.

Parameters:

	sk:传输控制块
*/
static void tcp_fin(struct sock *sk)
{
	struct tcp_sock *tp = tcp_sk(sk);

	inet_csk_schedule_ack(sk);

	sk->sk_shutdown |= RCV_SHUTDOWN;
	sock_set_flag(sk, SOCK_DONE);
\end{minted}

			首先,接收到FIN段后需要调度、发送ACK。

			其次,设置了传输控制块的套接口状态为\mintinline{C}{RCV_SHUTDOWN},表示服务器端不允许在接收数据。
	
			最后,设置传输控制块的\mintinline{C}{SOCK_DONE}标志,表示TCP会话结束。

			当然,接收到FIN字段的时候,TCP有可能并不是处于\mintinline{C}{TCP_ESTABLISHED}的,这里我们一并介绍了。

\begin{minted}[linenos]{C}
	switch (sk->sk_state) {
		case TCP_SYN_RECV:
		case TCP_ESTABLISHED:
			/* Move to CLOSE_WAIT */
			tcp_set_state(sk, TCP_CLOSE_WAIT);
			inet_csk(sk)->icsk_ack.pingpong = 1;			//what does it means pingpong?  to do.
			break;
\end{minted}

			如果TCP块处于\mintinline{C}{TCP_SYN_RECV}或者\mintinline{C}{TCP_ESTABLISHED}状态,接收到FIN之后,将状态设置为\mintinline{C}{CLOSE_WAIT},并确定延时发送ack。
\begin{minted}[linenos]{C}
		case TCP_CLOSE_WAIT:
		case TCP_CLOSING:
			/* Received a retransmission of the FIN, do
			 * nothing.
			 */
			break;
		case TCP_LAST_ACK:
			/* RFC793: Remain in the LAST-ACK state. */
			break;
\end{minted}

			在\mintinline{C}{TCP_CLOSE_WAIT}或者\mintinline{C}{TCP_CLOSING}状态下收到的FIN为重复收到的FIN,忽略。

			在\mintinline{C}{TCP_LAST_ACK}状态下正在等待最后的ACK字段,故而忽略。

\begin{minted}[linenos]{C}
		case TCP_FIN_WAIT1:
			/* This case occurs when a simultaneous close
			 * happens, we must ack the received FIN and
			 * enter the CLOSING state.
			 */
			tcp_send_ack(sk);
			tcp_set_state(sk, TCP_CLOSING);
			break;
\end{minted}

			显然,只有客户端和服务器端同时关闭的情况下,才会出现这种状态,而且此时双方必须都发送ACK字段,并且将状态置为\mintinline{C}{TCP_CLOSING}。

\begin{minted}[linenos]{C}
		case TCP_FIN_WAIT2:
			/* Received a FIN -- send ACK and enter TIME_WAIT. */
			tcp_send_ack(sk);
			tcp_time_wait(sk, TCP_TIME_WAIT, 0);
			break;
		default:
			/* Only TCP_LISTEN and TCP_CLOSE are left, in these
			 * cases we should never reach this piece of code.
			 */
			pr_err("%s: Impossible, sk->sk_state=%d\n",
				   __func__, sk->sk_state);
			break;
	}
\end{minted}

			在\mintinline{C}{TCP_FIN_WAIT2}状态下接收到FIN段,根据TCP状态转移图应该发送ACK响应服务器端。

			在其他状态,显然是不能收到FIN段的。

\begin{minted}[linenos]{C}
	/* It _is_ possible, that we have something out-of-order _after_ FIN.
	 * Probably, we should reset in this case. For now drop them.
	 */
	__skb_queue_purge(&tp->out_of_order_queue);
	if (tcp_is_sack(tp))
		tcp_sack_reset(&tp->rx_opt);
	sk_mem_reclaim(sk);

	if (!sock_flag(sk, SOCK_DEAD)) {
		sk->sk_state_change(sk);

		/* Do not send POLL_HUP for half duplex close. */
		if (sk->sk_shutdown == SHUTDOWN_MASK ||
		    sk->sk_state == TCP_CLOSE)
			sk_wake_async(sk, SOCK_WAKE_WAITD, POLL_HUP);
		else
			sk_wake_async(sk, SOCK_WAKE_WAITD, POLL_IN);
	}
}
\end{minted}

			清空接到的乱序队列上的段,再清除有关SACK的信息和标志,最后释放已接收队列中的段。

			如果此时套接口未处于DEAD状态,则唤醒等待该套接口的进程。如果在发送接收方向上都进行了关闭,或者此时传输控制块处于CLOSE状态,则唤醒异步等待该套接口的进程,通知它们该连接已经停止,否则通知它们连接可以进行写操作。
