\chapter{非核心函数分析}

\minitoc

\section{SKB}

\section{Inet}
    \subsection{\mintinline{C}{inet_hash_connect && __inet_hash_connect}}

        \subsubsection{\mintinline{C}{inet_hash_connect}}
\begin{minted}[linenos]{C}
/*
 * Bind a port for a connect operation and hash it.
 */
int inet_hash_connect(struct inet_timewait_death_row *death_row,
              struct sock *sk)
{
    u32 port_offset = 0;

    if (!inet_sk(sk)->inet_num)
        port_offset = inet_sk_port_offset(sk);
    return __inet_hash_connect(death_row, sk, port_offset,
                   __inet_check_established);
}
\end{minted}

        \subsubsection{\mintinline{C}{__inet_hash_connect}}     
\begin{minted}[linenos]{C}
int __inet_hash_connect(struct inet_timewait_death_row *death_row,
        struct sock *sk, u32 port_offset,
        int (*check_established)(struct inet_timewait_death_row *,
            struct sock *, __u16, struct inet_timewait_sock **))
{
    struct inet_hashinfo *hinfo = death_row->hashinfo;
    const unsigned short snum = inet_sk(sk)->inet_num;
    struct inet_bind_hashbucket *head;
    struct inet_bind_bucket *tb;
    int ret;
    struct net *net = sock_net(sk);

    if (!snum) {
        int i, remaining, low, high, port;
        static u32 hint;
        u32 offset = hint + port_offset;
        struct inet_timewait_sock *tw = NULL;

        inet_get_local_port_range(net, &low, &high);
        remaining = (high - low) + 1;

        /* By starting with offset being an even number,
         * we tend to leave about 50% of ports for other uses,
         * like bind(0).
         */
        offset &= ~1;

        local_bh_disable();
        for (i = 0; i < remaining; i++) {
            port = low + (i + offset) % remaining;
            if (inet_is_local_reserved_port(net, port))
                continue;
            head = &hinfo->bhash[inet_bhashfn(net, port,
                    hinfo->bhash_size)];
            spin_lock(&head->lock);

            /* Does not bother with rcv_saddr checks,
             * because the established check is already
             * unique enough.
             */
            inet_bind_bucket_for_each(tb, &head->chain) {
                if (net_eq(ib_net(tb), net) &&
                    tb->port == port) {
                    if (tb->fastreuse >= 0 ||
                        tb->fastreuseport >= 0)
                        goto next_port;
                    WARN_ON(hlist_empty(&tb->owners));
                    if (!check_established(death_row, sk,
                                port, &tw))
                        goto ok;
                    goto next_port;
                }
            }

            tb = inet_bind_bucket_create(hinfo->bind_bucket_cachep,
                    net, head, port);
            if (!tb) {
                spin_unlock(&head->lock);
                break;
            }
            tb->fastreuse = -1;
            tb->fastreuseport = -1;
            goto ok;

        next_port:
            spin_unlock(&head->lock);
        }
        local_bh_enable();

        return -EADDRNOTAVAIL;

ok:
        hint += (i + 2) & ~1;

        /* Head lock still held and bh's disabled */
        inet_bind_hash(sk, tb, port);
        if (sk_unhashed(sk)) {
            inet_sk(sk)->inet_sport = htons(port);
            inet_ehash_nolisten(sk, (struct sock *)tw);
        }
        if (tw)
            inet_twsk_bind_unhash(tw, hinfo);
        spin_unlock(&head->lock);

        if (tw)
            inet_twsk_deschedule_put(tw);

        ret = 0;
        goto out;
    }

    head = &hinfo->bhash[inet_bhashfn(net, snum, hinfo->bhash_size)];
    tb  = inet_csk(sk)->icsk_bind_hash;
    spin_lock_bh(&head->lock);
    if (sk_head(&tb->owners) == sk && !sk->sk_bind_node.next) {
        inet_ehash_nolisten(sk, NULL);
        spin_unlock_bh(&head->lock);
        return 0;
    } else {
        spin_unlock(&head->lock);
        /* No definite answer... Walk to established hash table */
        ret = check_established(death_row, sk, snum, NULL);
out:
        local_bh_enable();
        return ret;
    }
}
\end{minted}

\section{TCP层}


    \subsection{\mintinline{C}{tcp_sk}}

    \mintinline{c}{TCP_INC_STATS_BH}

    \mintinline{c}{rcu_read_unlock} 出现在\mintinline{c}{tcp_make_synack}中于MD5相关的部分。
    
    \mintinline{c}{net_xmit_eval}   定时器??

\subsection{\mintinline{c}{__tcp_push_pending_frames}}
\begin{minted}[linenos]{c}
/* 将等待在队列中的包全部发出。 */
void __tcp_push_pending_frames(struct sock *sk, unsigned int cur_mss,
                               int nonagle)
{
        /* 如果此时连接已经关闭了,那么直接返回。*/
        if (unlikely(sk->sk_state == TCP_CLOSE))
                return;

        /* 关闭nagle算法,将剩余的部分发送出去。 */
        if (tcp_write_xmit(sk, cur_mss, nonagle, 0,
                           sk_gfp_atomic(sk, GFP_ATOMIC)))
                tcp_check_probe_timer(sk);
}
\end{minted}

\subsection{\mintinline{c}{tcp_fin_time}}
计算等待接收FIN的超时时间。超时时间至少为$\frac{7}{2}$倍的rto。
\begin{minted}[linenos]{c}
static inline int tcp_fin_time(const struct sock *sk)
{
        int fin_timeout = tcp_sk(sk)->linger2 ? : sysctl_tcp_fin_timeout;
        const int rto = inet_csk(sk)->icsk_rto;

        if (fin_timeout < (rto << 2) - (rto >> 1))
                fin_timeout = (rto << 2) - (rto >> 1);

        return fin_timeout;
}
\end{minted}

\subsection{\mintinline{c}{tcp_done}}
\begin{minted}[linenos]{c}
/* 该函数用于完成关闭TCP连接,回收并清理相关资源。 */
void tcp_done(struct sock *sk)
{
        struct request_sock *req = tcp_sk(sk)->fastopen_rsk;

        /* 当套接字状态为SYN_SENT或SYN_RECV时,更新统计数据。 */
        if (sk->sk_state == TCP_SYN_SENT || sk->sk_state == TCP_SYN_RECV)
                TCP_INC_STATS_BH(sock_net(sk), TCP_MIB_ATTEMPTFAILS);

        /* 将连接状态设置为关闭,并清除定时器。 */
        tcp_set_state(sk, TCP_CLOSE);
        tcp_clear_xmit_timers(sk);
        /* 当启用了Fast Open时,移除fastopen请求 */
        if (req)
                reqsk_fastopen_remove(sk, req, false);

        sk->sk_shutdown = SHUTDOWN_MASK;

        /* 如果状态不为SOCK_DEAD,则唤醒等待着的进程。 */
        if (!sock_flag(sk, SOCK_DEAD))
                sk->sk_state_change(sk);
        else
                inet_csk_destroy_sock(sk);
}
\end{minted}

\subsection{tcp\_init\_nondata\_skb}
该函数提供了初始化不含数据的skb的功能。函数原型如下:
\begin{minted}[linenos]{c}
tcp_init_nondata_skb(struct sk_buff *skb, u32 seq, u8 flags);
\end{minted}

\begin{description}
  \item[skb] 待初始化的\mintinline{c}{sk_buff}。
  \item[seq] 序号
  \item[flags] 标志位
\end{description}

\begin{minted}[linenos]{c}
static void tcp_init_nondata_skb(struct sk_buff *skb, u32 seq, u8 flags)
{
        /* 设置校验码 */
        skb->ip_summed = CHECKSUM_PARTIAL;
        skb->csum = 0;

        /* 设置标志位 */
        TCP_SKB_CB(skb)->tcp_flags = flags;
        TCP_SKB_CB(skb)->sacked = 0;

        tcp_skb_pcount_set(skb, 1);

        /* 设置起始序号 */
        TCP_SKB_CB(skb)->seq = seq;
        if (flags & (TCPHDR_SYN | TCPHDR_FIN))
                seq++;
        TCP_SKB_CB(skb)->end_seq = seq;
}
\end{minted}

\subsection{before()和after()}
在一些需要判断序号前后的地方出现了\mintinline{c}{before()}和\mintinline{c}{after()}
这两个函数。这两个函数的定义如下
\begin{minted}[linenos]{c}
/* include/net/tcp.h
 * 比较两个无符号32位整数
 */
static inline bool before(__u32 seq1, __u32 seq2)
{
        return (__s32)(seq1-seq2) < 0;
}
#define after(seq2, seq1)       before(seq1, seq2)
\end{minted}
可以看到,这两个函数实际上就是将两个数直接相减。之所以要单独弄个函数应该是为了避免强制转型
造成影响。序号都是32位无符号整型。


