\section{TCP主动打开-客户}
        \subsection{基本流程}
        主动打开是通过connect系统调用来完成的。这一系统调用最终会调用传输层的\mintinline{c}{tcp_v4_connect}函数。
        \subsection{第一次握手——构造并发送SYN包}
\subsubsection{tcp\_v4\_connect}


\begin{minted}[linenos]{c}
/* This will initiate an outgoing connection. */
int tcp_v4_connect(struct sock *sk, struct sockaddr *uaddr, int addr_len)
{
        struct sockaddr_in *usin = (struct sockaddr_in *)uaddr;
        struct inet_sock *inet = inet_sk(sk);
        struct tcp_sock *tp = tcp_sk(sk);
        __be16 orig_sport, orig_dport;
        __be32 daddr, nexthop;
        struct flowi4 *fl4;
        struct rtable *rt;
        int err;
        struct ip_options_rcu *inet_opt;

        if (addr_len < sizeof(struct sockaddr_in))
                return -EINVAL;

        if (usin->sin_family != AF_INET)
                return -EAFNOSUPPORT;

        nexthop = daddr = usin->sin_addr.s_addr;
        inet_opt = rcu_dereference_protected(inet->inet_opt,
                                             sock_owned_by_user(sk));
        if (inet_opt && inet_opt->opt.srr) {
                if (!daddr)
                        return -EINVAL;
                nexthop = inet_opt->opt.faddr;
        }

        orig_sport = inet->inet_sport;
        orig_dport = usin->sin_port;
        fl4 = &inet->cork.fl.u.ip4;
        rt = ip_route_connect(fl4, nexthop, inet->inet_saddr,
                              RT_CONN_FLAGS(sk), sk->sk_bound_dev_if,
                              IPPROTO_TCP,
                              orig_sport, orig_dport, sk);
        if (IS_ERR(rt)) {
                err = PTR_ERR(rt);
                if (err == -ENETUNREACH)
                        IP_INC_STATS(sock_net(sk), IPSTATS_MIB_OUTNOROUTES);
                return err;
        }

        if (rt->rt_flags & (RTCF_MULTICAST | RTCF_BROADCAST)) {
                ip_rt_put(rt);
                return -ENETUNREACH;
        }

        if (!inet_opt || !inet_opt->opt.srr)
                daddr = fl4->daddr;

        if (!inet->inet_saddr)
                inet->inet_saddr = fl4->saddr;
        sk_rcv_saddr_set(sk, inet->inet_saddr);

        if (tp->rx_opt.ts_recent_stamp && inet->inet_daddr != daddr) {
                /* Reset inherited state */
                tp->rx_opt.ts_recent       = 0;
                tp->rx_opt.ts_recent_stamp = 0;
                if (likely(!tp->repair))
                        tp->write_seq      = 0;
        }

        if (tcp_death_row.sysctl_tw_recycle &&
            !tp->rx_opt.ts_recent_stamp && fl4->daddr == daddr)
                tcp_fetch_timewait_stamp(sk, &rt->dst);

        inet->inet_dport = usin->sin_port;
        sk_daddr_set(sk, daddr);

        inet_csk(sk)->icsk_ext_hdr_len = 0;
        if (inet_opt)
                inet_csk(sk)->icsk_ext_hdr_len = inet_opt->opt.optlen;

        tp->rx_opt.mss_clamp = TCP_MSS_DEFAULT;

        /* Socket identity is still unknown (sport may be zero).
         * However we set state to SYN-SENT and not releasing socket
         * lock select source port, enter ourselves into the hash tables and
         * complete initialization after this.
         */
        tcp_set_state(sk, TCP_SYN_SENT);
        err = inet_hash_connect(&tcp_death_row, sk);
        if (err)
                goto failure;

        sk_set_txhash(sk);

        rt = ip_route_newports(fl4, rt, orig_sport, orig_dport,
                               inet->inet_sport, inet->inet_dport, sk);
        if (IS_ERR(rt)) {
                err = PTR_ERR(rt);
                rt = NULL;
                goto failure;
        }
        /* OK, now commit destination to socket.  */
        sk->sk_gso_type = SKB_GSO_TCPV4;
        sk_setup_caps(sk, &rt->dst);

        if (!tp->write_seq && likely(!tp->repair))
                tp->write_seq = secure_tcp_sequence_number(inet->inet_saddr,
                                                           inet->inet_daddr,
                                                           inet->inet_sport,
                                                           usin->sin_port);

        inet->inet_id = tp->write_seq ^ jiffies;

        err = tcp_connect(sk);

        rt = NULL;
        if (err)
                goto failure;

        return 0;

failure:
        /*
         * This unhashes the socket and releases the local port,
         * if necessary.
         */
        tcp_set_state(sk, TCP_CLOSE);
        ip_rt_put(rt);
        sk->sk_route_caps = 0;
        inet->inet_dport = 0;
        return err;
}
\end{minted}

\subsubsection{tcp\_ connect}
上面的\mintinline{c}{tcp_v4_connect}会进行一系列的判断,之后真正构造SYN包的部分
被放置在了\mintinline{c}{tcp_connect}中。接下来,我们来分析这个函数。

\begin{minted}[linenos]{c}
/* 该函数用于构造并发送SYN包 */
int tcp_connect(struct sock *sk)
{
        struct tcp_sock *tp = tcp_sk(sk);
        struct sk_buff *buff;
        int err;

        /* 初始化tcp连接 */
        tcp_connect_init(sk);

        if (unlikely(tp->repair)) {
                /* 如果repair位被置1,那么结束TCP连接 */
                tcp_finish_connect(sk, NULL);
                return 0;
        }

        /* 分配一个sk_buff */
        buff = sk_stream_alloc_skb(sk, 0, sk->sk_allocation, true);
        if (unlikely(!buff))
                return -ENOBUFS;

        /* 初始化skb,并自增write_seq的值 */
        tcp_init_nondata_skb(buff, tp->write_seq++, TCPHDR_SYN);
        /* 设置时间戳 */
        tp->retrans_stamp = tcp_time_stamp;
        /* 将当前的sk_buff添加到发送队列中 */
        tcp_connect_queue_skb(sk, buff);
        /* ECN支持 */
        tcp_ecn_send_syn(sk, buff);

        /* 发送SYN包,这里同时还考虑了Fast Open的情况 */
        err = tp->fastopen_req ? tcp_send_syn_data(sk, buff) :
              tcp_transmit_skb(sk, buff, 1, sk->sk_allocation);
        if (err == -ECONNREFUSED)
                return err;

        /* We change tp->snd_nxt after the tcp_transmit_skb() call
         * in order to make this packet get counted in tcpOutSegs.
         */
        tp->snd_nxt = tp->write_seq;
        tp->pushed_seq = tp->write_seq;
        TCP_INC_STATS(sock_net(sk), TCP_MIB_ACTIVEOPENS);

        /* 设定超时重传定时器 */
        inet_csk_reset_xmit_timer(sk, ICSK_TIME_RETRANS,
                                  inet_csk(sk)->icsk_rto, TCP_RTO_MAX);
        return 0;
}
\end{minted}

\subsection{tcp\_transmit\_skb}

\subsubsection{函数依赖分析}

\begin{minted}[linenos]{c}
/* 为SYN包计算TCP选项,这个函数中计算出来的还不是最终的格式。
 */
static unsigned int tcp_syn_options(struct sock *sk, struct sk_buff *skb,
                                struct tcp_out_options *opts,
                                struct tcp_md5sig_key **md5);
/* 为已经建立连接的套接字计算TCP选项,这个函数中计算出来的还不是最终的格式。
 */
static unsigned int tcp_established_options(struct sock *sk, struct sk_buff *skb,
                                        struct tcp_out_options *opts,
                                        struct tcp_md5sig_key **md5);
/* 在skb中为头部流出空间。
 */
skb_push(skb, tcp_header_size);
/* 判断skb是否为一个纯ACK。这里把实现也放出来了。可以看到,纯ACK的包最显著
 * 的特点是其长度。Linux里通过判断长度直接快速判断出skb是否为一个纯ACK。
 */
static inline bool skb_is_tcp_pure_ack(const struct sk_buff *skb)
{
        return skb->truesize == 2;
}
/* 重置传输层的header的指针?
 */
static inline void skb_reset_transport_header(struct sk_buff *skb)
{
        skb->transport_header = skb->data - skb->head;
}
/* 选择发送窗口的大小
 */
tcp_select_window(sk);
tcp_urg_mode(tp);
before(tcb->seq, tp->snd_up);
tcp_options_write((__be32 *)(th + 1), tp, &opts);
tcp_event_ack_sent(sk, tcp_skb_pcount(skb));
tcp_event_data_sent(tp, sk);
queue_xmit();
tcp_enter_cwr(sk);
net_xmit_eval(err);
\end{minted}

\subsection{tcp\_select\_window(struct sk\_buff *skb)}
这个函数的作用是选择一个新的窗口大小以用于更新\mintinline{c}{tcp_sock}。
返回的结果根据RFC1323进行了缩放。

\subsubsection{RFC1323——高性能TCP扩展(TCP Extensions for High Performance)}
这个RFC主要是在考虑高带宽高延迟网络下如何提升TCP的性能。就好像一个又粗又长的管道,
如果想要管道的利用率高,就要尽可能地把管道填满。但是TCP能够同时发送的东西的上限是
受到发送窗口的限制的。超过了窗口大小,就必须等待ACK确认才可以继续发送。

然而,在TCP头部中,只有16位的一个域用于说明窗口大小。也就是说,窗口大小最大只能
达到$2^{16}=64K字节$。为了解决这一问题,RFC1323新增了一个TCP选项,用于放大窗口的
大小。该选项的值代表将原窗口大小放大2的幂倍。

个人认为这个设计很有好。采用2的幂来缩放可以很大程度地扩展窗口的大小,因为2的幂
增长得很快。而且可以通过位移运算来实现缩放,性能上也很好。

\subsubsection{代码分析}
\begin{minted}[linenos]{c}
static u16 tcp_select_window(struct sock *sk)
{
        struct tcp_sock *tp = tcp_sk(sk);
        u32 old_win = tp->rcv_wnd;
        u32 cur_win = tcp_receive_window(tp);
        u32 new_win = __tcp_select_window(sk);
        /* old_win是接收方窗口的大小。
         * cur_win当前的接收窗口大小。
         * new_win是新选择出来的窗口大小。
         */

        /* 当新窗口的大小小于当前窗口的大小时,不能缩减窗口大小。
         * 这是IEEE强烈不建议的一种行为。
         */
        if (new_win < cur_win) {
                /* Danger Will Robinson!
                 * Don't update rcv_wup/rcv_wnd here or else
                 * we will not be able to advertise a zero
                 * window in time.  --DaveM
                 *
                 * Relax Will Robinson.
                 */
                if (new_win == 0)
                        NET_INC_STATS(sock_net(sk),
                                      LINUX_MIB_TCPWANTZEROWINDOWADV);
                /* 当计算出来的新窗口小于当前窗口时,将新窗口设置为大于cur_win
                 * 的1<<tp->rx_opt.rcv_wscale的整数倍。
                 */
                new_win = ALIGN(cur_win, 1 << tp->rx_opt.rcv_wscale);
        }
        /* 将当前的接收窗口设置为新的窗口大小。*/
        tp->rcv_wnd = new_win;
        tp->rcv_wup = tp->rcv_nxt;

        /* 判断当前窗口未越界。*/
        if (!tp->rx_opt.rcv_wscale && sysctl_tcp_workaround_signed_windows)
                new_win = min(new_win, MAX_TCP_WINDOW);
        else
                new_win = min(new_win, (65535U << tp->rx_opt.rcv_wscale));

        /* RFC1323 缩放窗口大小。这里之所以是右移,是因为此时的new_win是
         * 窗口的真正大小。所以返回时需要返回正常的可以放在16位整型中的窗口大小。
         * 所以需要右移。
         */
        new_win >>= tp->rx_opt.rcv_wscale;

        /* If we advertise zero window, disable fast path. */
        if (new_win == 0) {
                tp->pred_flags = 0;
                if (old_win)
                        NET_INC_STATS(sock_net(sk),
                                      LINUX_MIB_TCPTOZEROWINDOWADV);
        } else if (old_win == 0) {
                NET_INC_STATS(sock_net(sk), LINUX_MIB_TCPFROMZEROWINDOWADV);
        }

        return new_win;
}
\end{minted}

在这个过程中,还调用了\mintinline{c}{__tcp_select_window(sk)}来计算新的窗口大小。
该函数会尝试增加窗口的大小,但是有两个限制条件:

\begin{enumerate}
  \item 窗口不能收缩(RFC793)
  \item 每个socket所能使用的内存是有限制的。
\end{enumerate}

RFC 1122中说:
\begin{quote}
"the suggested [SWS] avoidance algorithm for the receiver is to keep
RECV.NEXT + RCV.WIN fixed until:
RCV.BUFF - RCV.USER - RCV.WINDOW >= min(1/2 RCV.BUFF, MSS)"

推荐的用于接收方的糊涂窗口综合症的避免算法是保持recv.next+rcv.win不变,直到:
RCV.BUFF - RCV.USER - RCV.WINDOW >= min(1/2 RCV.BUFF, MSS)
\end{quote}

换句话说,就是除非缓存的大小多出来至少一个MSS那么多字节,否则不要增长窗口右边界
的大小。

然而,根据Linux注释中的说法,被推荐的这个算法会破坏头预测(header prediction),
因为头预测会假定\mintinline{c}{th->window}不变。严格地说,
保持\mintinline{c}{th->window}固定不变会违背接收方的用于防止糊涂窗口综合症的准则。
在这种规则下,一个单字节的包的流会引发窗口的右边界总是提前一个字节。
当然,如果发送方实现了预防糊涂窗口综合症的方法,那么就不会出现问题。

Linux的TCP部分的作者们参考了BSD的实现方法。BSD在这方面的做法是是,
如果空闲空间小于最大可用空间的$\frac{1}{4}$,且空闲空间
小于mss的$\frac{1}{2}$,那么就把窗口设置为0。否则,只是单纯地阻止窗口缩小,
或者阻止窗口大于最大可表示的范围(the largest representable value)。
BSD的方法似乎“意外地”使得窗口基本上都是MSS的整倍数。且很多情况下窗口大小都是
固定不变的。因此,Linux采用强制窗口为MSS的整倍数,以获得相似的行为。

\begin{minted}[linenos]{c}
u32 __tcp_select_window(struct sock *sk)
{
        struct inet_connection_sock *icsk = inet_csk(sk);
        struct tcp_sock *tp = tcp_sk(sk);
        int mss = icsk->icsk_ack.rcv_mss;
        int free_space = tcp_space(sk);
        int allowed_space = tcp_full_space(sk);
        int full_space = min_t(int, tp->window_clamp, allowed_space);
        int window;

        /* 如果mss超过了总共的空间大小,那么把mss限制在允许的空间范围内。 */
        if (mss > full_space)
                mss = full_space;

        if (free_space < (full_space >> 1)) {
                /* 当空闲空间小于允许空间的一半时。 */
                icsk->icsk_ack.quick = 0;

                if (tcp_under_memory_pressure(sk))
                        tp->rcv_ssthresh = min(tp->rcv_ssthresh,
                                               4U * tp->advmss);

                /* free_space有可能成为新的窗口的大小,因此,需要考虑
                 * 窗口扩展的影响。
                 */
                free_space = round_down(free_space, 1 << tp->rx_opt.rcv_wscale);

                /* 如果空闲空间小于mss的大小,或者低于最大允许空间的的1/16,那么,
                 * 返回0窗口。否则,tcp_clamp_window()会增长接收缓存到tcp_rmem[2]。
                 * 新进入的数据会由于内醋限制而被丢弃。对于较大的窗口,单纯地探测mss的
                 * 大小以宣告0窗口有些太晚了(可能会超过限制)。
                 */
                if (free_space < (allowed_space >> 4) || free_space < mss)
                        return 0;
        }

        if (free_space > tp->rcv_ssthresh)
                free_space = tp->rcv_ssthresh;

        /* 这里处理一个例外情况,就是如果开启了窗口缩放,那么就没法对齐mss了。
         * 所以就保持窗口是对齐2的幂的。
         */
        window = tp->rcv_wnd;
        if (tp->rx_opt.rcv_wscale) {
                window = free_space;

                /* Advertise enough space so that it won't get scaled away.
                 * Import case: prevent zero window announcement if
                 * 1<<rcv_wscale > mss.
                 */
                if (((window >> tp->rx_opt.rcv_wscale) << tp->rx_opt.rcv_wscale) != window)
                        window = (((window >> tp->rx_opt.rcv_wscale) + 1)
                                  << tp->rx_opt.rcv_wscale);
        } else {
                /* 如果内存条件允许,那么就把窗口设置为mss的整倍数。
                 * 或者如果free_space > 当前窗口大小加上全部允许的空间的一半,
                 * 那么,就将窗口大小设置为free_space
                 */
                if (window <= free_space - mss || window > free_space)
                        window = (free_space / mss) * mss;
                else if (mss == full_space &&
                         free_space > window + (full_space >> 1))
                        window = free_space;
        }

        return window;
}
\end{minted}
 
