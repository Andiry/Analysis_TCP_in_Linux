\chapter{准备部分}

\minitoc

\section{用户层TCP}

用户层的TCP编程模型大致如下,对于服务端,调用listen监听端口,
之后接受客户端的请求,然后就可以收发数据了。结束时,关闭socket。

\begin{minted}[linenos]{c}
// Server
socket(...,SOCK_STREAM,0);
bind(...,&server_address, ...);
listen(...);
accept(..., &client_address, ...);
recv(..., &clientaddr, ...);
close(...);
\end{minted}

对于客户端,则调用connect连接服务端,之后便可以收发数据。
最后关闭socket。

\begin{minted}[linenos]{c}
socket(...,SOCK_STREAM,0);
connect();
send(...,&server_address,...);
\end{minted}

那么根据我们的需求,我们着重照顾连接的建立、关闭和封包的收发过程。

\section{探寻tcp\_prot,地图get\textasciitilde{}}

一般游戏的主角手中,都会有一张万能的地图。为了搞定TCP,我们自然也是需要
一张地图的,要不连该去找那个函数看都不知道。很有幸,在\mintinline[linenos]{text}{tcp_ipv4.c}中,
\mintinline[linenos]{text}{tcp_prot}定义了\mintinline[linenos]{text}{tcp}的各个接口。

\mintinline[linenos]{text}{tcp_prot}的类型为\mintinline[linenos]{text}{struct proto},是这个结构体是为了抽象各种不同的协议的
差异性而存在的。类似面向对象中所说的接口(Interface)的概念。这里,我们仅
保留我们关系的部分。

\begin{minted}[linenos]{c}
struct proto tcp_prot = {
        .name                   = "TCP",
        .owner                  = THIS_MODULE,
        .close                  = tcp_close,
        .connect                = tcp_v4_connect,
        .disconnect             = tcp_disconnect,
        .accept                 = inet_csk_accept,
        .destroy                = tcp_v4_destroy_sock,
        .shutdown               = tcp_shutdown,
        .setsockopt             = tcp_setsockopt,
        .getsockopt             = tcp_getsockopt,
        .recvmsg                = tcp_recvmsg,
        .sendmsg                = tcp_sendmsg,
        .sendpage               = tcp_sendpage,
        .backlog_rcv            = tcp_v4_do_rcv,
        .get_port               = inet_csk_get_port,
        .twsk_prot              = &tcp_timewait_sock_ops,
        .rsk_prot               = &tcp_request_sock_ops,
};
\end{minted}

通过名字,我大致筛选出来了这些函数,初步判断这些函数与实验所关心的功能相关。
对着这张``地图'',就可以顺藤摸瓜,找出些路径了。

先根据参考书《Linux内核源码剖析------TCP/IP实现》中给出的流程图,
找出所有和需求相关的部分。

首先找三次握手相关的部分:从客户端的角度,发起连接需要调用\mintinline[linenos]{text}{tcp_v4_connect},
该函数会进一步调用\mintinline[linenos]{text}{tcp_connect},在这个函数中,会调用\mintinline[linenos]{text}{tcp_send_syn_data}
发送SYN报文,并设定超时计时器。第二次握手相关的接收代码在\mintinline[linenos]{text}{tcp_rcv_state_process}中,
该函数实现了除\mintinline[linenos]{text}{ESTABLISHED}和\mintinline[linenos]{text}{TIME_WAIT}之外所有状态下的接收处理。
\mintinline[linenos]{text}{tcp_send_ack}函数实现了发送ACK报文。从服务端的角度,则还需实现\mintinline[linenos]{text}{listen}调用和
\mintinline[linenos]{text}{accept}调用。二者都是服务端建立连接所需要的部分。

封包的封装发送部分,所对应的函数是\mintinline[linenos]{text}{tcp_sendmsg},实现对数据的复制、切割和发送。
TCP的重传接口为\mintinline[linenos]{text}{tcp_retransmit_skb},这里尚有疑问,因为这个函数是负责处理重传的,
而不是判断是否应当重传的。所以并不明确到底是否该重新实现这一部分。

TCP封包的接收在\mintinline[linenos]{text}{tcp_rcv_established}函数中,根据目前有限的资料看,TCP的滑动窗口机制应该
在这一部分,更细节的内容待确认。

目前,待重新实现的函数列表是:

\begin{itemize}
\item
  tcp\_transmit\_skb
\item
  tcp\_rcv\_state\_process
\item
  tcp\_connect
\item
  tcp\_rcv\_synsent\_state\_process
\item
  tcp\_rcv\_established
\item
  tcp\_send\_ack
\item
  tcp\_sendmsg
\item
  tcp\_retransmit\_skb(存疑)
\item
  tcp\_rcv\_established
\item
  accept和listen(待详细调查)
\item
  更多需要等进一步仔细阅读后再做决定
\end{itemize}
