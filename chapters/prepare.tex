\chapter{准备部分}

\minitoc

\section{用户层TCP}

用户层的TCP编程模型大致如下,对于服务端,调用listen监听端口,
之后接受客户端的请求,然后就可以收发数据了。结束时,关闭socket。

\begin{minted}[linenos]{c}
// Server
socket(...,SOCK_STREAM,0);
bind(...,&server_address, ...);
listen(...);
accept(..., &client_address, ...);
recv(..., &clientaddr, ...);
close(...);
\end{minted}

对于客户端,则调用connect连接服务端,之后便可以收发数据。
最后关闭socket。

\begin{minted}[linenos]{c}
socket(...,SOCK_STREAM,0);
connect();
send(...,&server_address,...);
\end{minted}

那么根据我们的需求,我们着重照顾连接的建立、关闭和封包的收发过程。

\section{探寻tcp\_prot,地图get\textasciitilde{}}

一般游戏的主角手中,都会有一张万能的地图。为了搞定TCP,我们自然也是需要
一张地图的,要不连该去找那个函数看都不知道。很有幸,在\mintinline[linenos]{text}{tcp_ipv4.c}中,
\mintinline[linenos]{text}{tcp_prot}定义了\mintinline[linenos]{text}{tcp}的各个接口。

\mintinline[linenos]{text}{tcp_prot}的类型为\mintinline[linenos]{text}{struct proto},是这个结构体是为了抽象各种不同的协议的
差异性而存在的。类似面向对象中所说的接口(Interface)的概念。这里,我们仅
保留我们关系的部分。

\begin{minted}[linenos]{c}
struct proto tcp_prot = {
        .name                   = "TCP",
        .owner                  = THIS_MODULE,
        .close                  = tcp_close,
        .connect                = tcp_v4_connect,
        .disconnect             = tcp_disconnect,
        .accept                 = inet_csk_accept,
        .destroy                = tcp_v4_destroy_sock,
        .shutdown               = tcp_shutdown,
        .setsockopt             = tcp_setsockopt,
        .getsockopt             = tcp_getsockopt,
        .recvmsg                = tcp_recvmsg,
        .sendmsg                = tcp_sendmsg,
        .sendpage               = tcp_sendpage,
        .backlog_rcv            = tcp_v4_do_rcv,
        .get_port               = inet_csk_get_port,
        .twsk_prot              = &tcp_timewait_sock_ops,
        .rsk_prot               = &tcp_request_sock_ops,
};
\end{minted}

通过名字,我大致筛选出来了这些函数,初步判断这些函数与实验所关心的功能相关。
对着这张``地图'',就可以顺藤摸瓜,找出些路径了。

先根据参考书《Linux内核源码剖析------TCP/IP实现》中给出的流程图,
找出所有和需求相关的部分。

首先找三次握手相关的部分:从客户端的角度,发起连接需要调用\mintinline[linenos]{text}{tcp_v4_connect},
该函数会进一步调用\mintinline[linenos]{text}{tcp_connect},在这个函数中,会调用\mintinline[linenos]{text}{tcp_send_syn_data}
发送SYN报文,并设定超时计时器。第二次握手相关的接收代码在\mintinline[linenos]{text}{tcp_rcv_state_process}中,
该函数实现了除\mintinline[linenos]{text}{ESTABLISHED}和\mintinline[linenos]{text}{TIME_WAIT}之外所有状态下的接收处理。
\mintinline[linenos]{text}{tcp_send_ack}函数实现了发送ACK报文。从服务端的角度,则还需实现\mintinline[linenos]{text}{listen}调用和
\mintinline[linenos]{text}{accept}调用。二者都是服务端建立连接所需要的部分。

封包的封装发送部分,所对应的函数是\mintinline[linenos]{text}{tcp_sendmsg},实现对数据的复制、切割和发送。
TCP的重传接口为\mintinline[linenos]{text}{tcp_retransmit_skb},这里尚有疑问,因为这个函数是负责处理重传的,
而不是判断是否应当重传的。所以并不明确到底是否该重新实现这一部分。

TCP封包的接收在\mintinline[linenos]{text}{tcp_rcv_established}函数中,根据目前有限的资料看,TCP的滑动窗口机制应该
在这一部分,更细节的内容待确认。

\begin{itemize}
\item
  tcp\_transmit\_skb
\item
  tcp\_rcv\_state\_process
\item
  tcp\_connect
\item
  tcp\_rcv\_synsent\_state\_process
\item
  tcp\_rcv\_established
\item
  tcp\_send\_ack
\item
  tcp\_sendmsg
\item
  tcp\_retransmit\_skb
\item
  tcp\_rcv\_established
\end{itemize}

\section{RFC}
\label{sec:rfc}

在分析TCP的过程中会遇到很多RFC,在这里,我们将可能会碰到的RFC罗列出来,
并进行一定的讨论,便于后面的分析。

\subsection{RFC793——Transmission Control Protocol}
\label{subsec:rfc793}

该RFC正是定义了TCP协议的那份RFC。在该RFC中,可以查到TCP的很多细节,帮助后续的代码分析。

\subsubsection{TCP状态图}
在RFC793中,给出了TCP协议的状态图,图中的TCB代表TCP控制块。原图如下所示:
\begin{minted}[linenos]{text}
                              +---------+ ---------\      active OPEN  
                              |  CLOSED |            \    -----------  
                              +---------+<---------\   \   create TCB  
                                |     ^              \   \  snd SYN    
                   passive OPEN |     |   CLOSE        \   \           
                   ------------ |     | ----------       \   \         
                    create TCB  |     | delete TCB         \   \       
                                V     |                      \   \     
                              +---------+            CLOSE    |    \   
                              |  LISTEN |          ---------- |     |  
                              +---------+          delete TCB |     |  
                   rcv SYN      |     |     SEND              |     |  
                  -----------   |     |    -------            |     V  
 +---------+      snd SYN,ACK  /       \   snd SYN          +---------+
 |         |<-----------------           ------------------>|         |
 |   SYN   |                    rcv SYN                     |   SYN   |
 |   RCVD  |<-----------------------------------------------|   SENT  |
 |         |                    snd ACK                     |         |
 |         |------------------           -------------------|         |
 +---------+   rcv ACK of SYN  \       /  rcv SYN,ACK       +---------+
   |           --------------   |     |   -----------                  
   |                  x         |     |     snd ACK                    
   |                            V     V                                
   |  CLOSE                   +---------+                              
   | -------                  |  ESTAB  |                              
   | snd FIN                  +---------+                              
   |                   CLOSE    |     |    rcv FIN                     
   V                  -------   |     |    -------                     
 +---------+          snd FIN  /       \   snd ACK          +---------+
 |  FIN    |<-----------------           ------------------>|  CLOSE  |
 | WAIT-1  |------------------                              |   WAIT  |
 +---------+          rcv FIN  \                            +---------+
   | rcv ACK of FIN   -------   |                            CLOSE  |  
   | --------------   snd ACK   |                           ------- |  
   V        x                   V                           snd FIN V  
 +---------+                  +---------+                   +---------+
 |FINWAIT-2|                  | CLOSING |                   | LAST-ACK|
 +---------+                  +---------+                   +---------+
   |                rcv ACK of FIN |                 rcv ACK of FIN |  
   |  rcv FIN       -------------- |    Timeout=2MSL -------------- |  
   |  -------              x       V    ------------        x       V  
    \ snd ACK                 +---------+delete TCB         +---------+
     ------------------------>|TIME WAIT|------------------>| CLOSED  |
                              +---------+                   +---------+

                      TCP Connection State Diagram
\end{minted}
这张图对于后面的分析有很强的指导意义。

连接部分分为了主动连接和被动连接。主动连接是指客户端从CLOSED状态主动发出连接请求,
进入SYN-SENT状态,之后收到服务端的SYN+ACK包,进入ESTAB状态(即连接建立状态),
然后回复ACK包,完成三次握手。这一部分的代码我们将在\ref{sec:tcp_connect_client}
中进行详细分析。被动连接是从listen状态开始,监听端口。随后收到SYN包,进入SYN-RCVD状态,
同时发送SYN+ACK包,最后,收到ACK后,进入ESTAB状态,完成被动连接的三次握手过程。
这一部分的详细讨论在\ref{sec:tcp_server_connect}中完成。

连接终止的部分也被分为了两部分进行实现,主动终止和被动终止。主动终止是上图中从ESTAB状态
主动终止连接,发送FIN包的过程。可以看到,主动终止又分为两种情况,一种是FIN发出后,收到了
发来的FIN(即通信双方同时主动关闭连接),此时转入CLOSING状态并发送ACK包。收到ACK后,
进入TIME WAIT状态。另一种是收到了ACK包,转入FINWAIT-2状态,最后收到FIN后发送ACK,
完成四次握手,进入TIME WAIT状态。最后等数据发送完或者超时后,删除TCB,进入CLOSED状态。
被动终止则是接收到FIN包后,发送了ACK包,进入CLOSE WAIT状态。之后,当这一端的数据
也发送完成后,发送FIN包,进入LAST-ACK状态,接收到ACK后,进入CLOSED状态。

\subsubsection{TCP头部格式}
\label{subsubsec:tcp_header_format}
RFC793中,对于TCP头部格式的描述摘录如下:
\begin{minted}[linenos]{text}
    0                   1                   2                   3   
    0 1 2 3 4 5 6 7 8 9 0 1 2 3 4 5 6 7 8 9 0 1 2 3 4 5 6 7 8 9 0 1 
   +-+-+-+-+-+-+-+-+-+-+-+-+-+-+-+-+-+-+-+-+-+-+-+-+-+-+-+-+-+-+-+-+
   |          Source Port          |       Destination Port        |
   +-+-+-+-+-+-+-+-+-+-+-+-+-+-+-+-+-+-+-+-+-+-+-+-+-+-+-+-+-+-+-+-+
   |                        Sequence Number                        |
   +-+-+-+-+-+-+-+-+-+-+-+-+-+-+-+-+-+-+-+-+-+-+-+-+-+-+-+-+-+-+-+-+
   |                    Acknowledgment Number                      |
   +-+-+-+-+-+-+-+-+-+-+-+-+-+-+-+-+-+-+-+-+-+-+-+-+-+-+-+-+-+-+-+-+
   |  Data |           |U|A|P|R|S|F|                               |
   | Offset| Reserved  |R|C|S|S|Y|I|            Window             |
   |       |           |G|K|H|T|N|N|                               |
   +-+-+-+-+-+-+-+-+-+-+-+-+-+-+-+-+-+-+-+-+-+-+-+-+-+-+-+-+-+-+-+-+
   |           Checksum            |         Urgent Pointer        |
   +-+-+-+-+-+-+-+-+-+-+-+-+-+-+-+-+-+-+-+-+-+-+-+-+-+-+-+-+-+-+-+-+
   |                    Options                    |    Padding    |
   +-+-+-+-+-+-+-+-+-+-+-+-+-+-+-+-+-+-+-+-+-+-+-+-+-+-+-+-+-+-+-+-+
   |                             data                              |
   +-+-+-+-+-+-+-+-+-+-+-+-+-+-+-+-+-+-+-+-+-+-+-+-+-+-+-+-+-+-+-+-+

                            TCP Header Format

          Note that one tick mark represents one bit position.
\end{minted}

这张图可以很方便地读出各个位占多长,它上面的标识是十进制的,很容易读。
这里我们挑出我们比较关心的Options字段来解读。因为很多TCP的扩展都是通过新增
选项来实现的。

    选项总是在TCP头部的最后,且长度是 8 位的整数倍。全部选项都被包括在
校验和中。选项可从任何字节边界开始。选项的格式有 2 种情况\textbf{有待补充}:
\begin{enumerate}
\item 单独的一个字节,代表选项的类型例如:

\begin{minted}[linenos]{text}
  End of Option List
  +--------+
  |00000000| 
  +--------+ 
  Kind=0
\end{minted}

\item 第一个字节代表选项的类型,紧跟着的一个字节代表选项的长度,后面跟着
选项的数据。例如:

\begin{minted}[linenos]{text}
  Maximum Segment Size
  +--------+--------+---------+--------+ 
  |00000010|00000100| max seg size     | 
  +--------+--------+---------+--------+ 
    Kind=2  Length=4
\end{minted}

\end{enumerate}

\subsection{RFC1323——TCP Extensions for High Performance}
\label{subsec:rfc1323}

\subsubsection{简介}
\label{subsubsec:rfc1323_introduce}

这个RFC主要是考虑高带宽高延迟网络下如何提升TCP的性能。该RFC定义了新的TCP选项,
以实现窗口缩放(window scaled)和时间戳(timestamp)。这里的时间戳可以用于实现
两个机制:RTTM(Round Trip Time Measurement)和PAWS(Protect Against Wrapped Sequences)。

在RFC1323中提出,在这类高带宽高延迟网络下,有三个主要的影响TCP性能的因素:
\begin{description}
  \item[窗口尺寸限制] 在TCP头部中,只有16位的一个域用于说明窗口大小。也就是说,
    窗口大小最大只能达到$2^{16}=64K$字节。解决这一问题的方案是增加一个窗口缩放
    选项,我们会在\ref{subsubsec:window_scale}中进一步讨论。
  \item[丢包后的恢复] 丢包会导致TCP重新进入慢启动状态,导致数据的流水线断流。
    在引入了快重传和快恢复后,可以解决丢包率为一个窗口中丢一个包的情况下的问题。
    但是在引入了窗口缩放以后,由于窗口的扩大,丢包的概率也随之增加。很容易使TCP进入
    到慢启动状态,影响网络性能。为了解决这一问题,需要引入SACK机制,但在这个RFC中,
    不讨论SACK相关的问题。
  \item[往返时间度量] RTO(Retransmission timeout)是TCP性能的一个很基础的参数。
    在RFC1323中介绍了一种名为RTTM的机制,利用一个新的名为Timestamps的选项来对时间
    进行进一步的统计。
\end{description}

\subsubsection{窗口缩放(Window Scale)}
\label{subsubsec:window_scale}
这一扩展将原有的TCP窗口扩展到32位。而根据RFC793中的定义,TCP头部描述窗口大小的域仅有16位。
为了将其扩展为32位,该扩展定义了一个新的选项用于表示缩放因子。这一选项仅会出现在SYN段,
此后,所有通过该连接的通信,其窗口大小都会受到这一选项的影响。

该选项的格式为:
\begin{minted}[linenos]{text}
  +---------+---------+---------+ 
  | Kind=3  |Length=3 |shift.cnt| 
  +---------+---------+---------+
\end{minted}

\mintinline{text}{kind}域为3,\mintinline{text}{length}域为3,后面跟着3个字节的
\mintinline{text}{shift.cnt},代表缩放因子。TCP的Options域的选项的格式在
\ref{subsubsec:tcp_header_format}中已有说明。这里采用的是第二种格式。

在启用了窗口缩放以后,TCP头部中的接收窗口大小,就变为了真实的接收窗口大小右移
\mintinline{text}{shift.cnt}的值。RFC中对于该选项的实现的建议是在传输控制块中,
按照32位整型来存储所有的窗口值,包括发送窗口、接受窗口和拥塞窗口。

作为接收方,每当收到一个段时(除SYN段外),通过将发来的窗口值左移来得到正确的值。
\begin{minted}[linenos]{c}
SND.WND = SEG.WND << Snd.Wind.Scale
\end{minted}

作为发送方则每次在发包前,将发送窗口的值右移,然后再封装在封包中。
\begin{minted}[linenos]{c}
SEG.WND = RCV.WND >> Rcv.Wind.Scale
\end{minted}

\subsubsection{PAWS(Protect Against Wrapped Sequence Numbers)}
PAWS是一个用于防止旧的重复封包带来的问题的机制。它采用了TCP中的Timestamps选项。
该选项的格式为:
\begin{minted}[linenos]{text}
+-------+-------+---------------------+---------------------+ 
|Kind=8 |  10   |   TS Value (TSval)  |TS Echo Reply (TSecr)| 
+-------+-------+---------------------+---------------------+
    1       1              4                      4
\end{minted}

TSval(Timestamp Value)包含了TCP发送方的时钟的当前值。如果ACK位被设置了的话,
TSecr(Timestamp Echo Reply)会包含一个由对方发送过来的最近的时钟值。

PAWS的算法流程为:
\begin{enumerate}
  \item 如果当前到达的段SEG含有Timestamps选项,且SEG.TSval < TS.Recent且
该TS.Recent是有效的,那么则认为当前到达的分段是不可被接受的,需要丢弃掉。
  \item 如果段超过了窗口的范围,则丢弃它(与正常的TCP处理相同)
  \item 如果满足SEG.SEQ$\leqslant$Last.ACK.sent(最后回复的ACK包中的时间戳),
    那么,将SEG的时间戳记录为TS.Recent。
  \item 如果SEG是正常按顺序到达的,那么正常地接收它。
  \item 其他情况下,将该段视作正常的在窗口中,但顺序不正确的TCP段对待。
\end{enumerate}

\subsection{RFC3168——The Addition of Explicit Congestion Notification (ECN) to IP}
\label{subsec:rfc3168}
该RFC为TCP/IP网络引入了一种新的用于处理网络拥塞问题的方法。在以往的TCP中,
人们假定网络是一个不透明的黑盒。而发送方通过丢包的情况来推测路由器处发生了拥塞。
这种拥塞控制方法对于交互式的应用效果并不理想,且可能对吞吐量造成负面影响。
因为此时网络可能已经计入到了拥塞状态(路由器的缓存满了,因此发生了丢包)。
为了解决这一问题,人们已经引入了主动队列管理算法(AQM)。允许路由器在拥塞发生的早期,
即队列快要满了的时候,就通过主动丢包的方式,告知发送端要减慢发送速率。
然而,主动丢包会引起包的重传,这会降低网络的性能。因此,RFC3168引入了ECN(显式拥塞
通知)机制。

\subsubsection{ECN}
\label{subsubsec:ecn}
在引入ECN机制后,AQM通知发送端的方式不再局限于主动丢包,而是可以通过IP头部的
Congestion Experienced(CE)来通知支持ECN机制的发送端发生了拥塞。这样,接收端
可以不必通过丢包来实现AQM,减少了发送端因丢包而产生的性能下降。

ECN在IP包中的域如下所示:
\begin{minted}[linenos]{text}
+-----+-----+ 
| ECN FIELD | 
+-----+-----+ 
ECT  CE  [Obsolete] RFC 2481 names for the ECN bits. 
 0   0   Not-ECT 
 0   1   ECT(1) 
 1   0   ECT(0) 
 1   1   CE
\end{minted}

Not-ECT代表该包没有采用ECN机制。ECT(1)和ECT(0)均代表该设备支持ECN机制。
CE用于路由器向发送端表明网络是否正在经历拥塞状态。

\subsubsection{TCP中的ECN}
在TCP协议中,利用TCP头部的保留位为ECN提供了支持。这里规定了两个新的位:ECE和CWR。
ECE用于在三次握手中协商手否启用ECN功能。CWR用于让接收端决定合适可以停止设置ECE标志。
增加ECN支持后的TCP头部示意图如下:
\begin{minted}[linenos]{text}
0     1   2   3   4   5   6   7   8   9  10  11  12  13  14  15 
+---+---+---+---+---+---+---+---+---+---+---+---+---+---+---+---+ 
|               |               | C | E | U | A | P | R | S | F | 
| Header Length |   Reserved    | W | C | R | C | S | S | Y | I | 
|               |               | R | E | G | K | H | T | N | N | 
+---+---+---+---+---+---+---+---+---+---+---+---+---+---+---+---+
\end{minted}

对于一个TCP连接,一个典型的基于ECN的序列如下:
\begin{enumerate}
  \item 发送方在发送的包中设置ECT位,表明发送端支持ECN。
  \item 在该包经历一个支持ECN的路由器时,该路由器发现了拥塞,准备主动丢包。
    此时,它发现正要丢弃的包支持ECN。它会放弃丢包,而是设置IP头部的CE位,
    表明经历了拥塞。
  \item 接收方收到设置了CE位的包后,在回复的ACK中,设置ECE标记。
  \item 发送端收到了一个设定了ECE标记的ACK包,进入和发生丢包一样的拥塞控制状态。
  \item 发送端在下一个要发送的包中设定CWR位,以通知接收方它已经对ECE位进行了相应处理。
\end{enumerate}

通过这种方式,就可以在经历网络拥塞时减少丢包的情况,并通知客户端网络的拥塞状态。

\subsection{RFC7413——TCP Fast Open(Draft)}
\label{subsec:rfc7413}

RFC7413目前还处于草案状态,它引入了一种试验性的特性:允许在三次握手阶段的SYN和SYN-ACK包中
携带数据。相较于标准的三次握手,引入TFO(TCP Fast Open)机制可以节省一个RTT的时间。
该RFC由Google提交,并在Linux和Chrome中实现了对该功能的支持。此后,越来越多的软件也
支持了该功能。

\subsubsection{概要}
\label{subsubsec:rfc7413-overview}
TFO中,最核心的一个部分是Fast Open Cookie。这个Cookie是由服务器生成的,在初次
进行常规的TCP连接时,由客户端向服务器请求,之后,则可利用这个Cookie在后续的TCP
连接中在三次握手阶段交换数据。

请求Fast Open Cookie的过程为:
\begin{enumerate}
  \item 客户端发送一个带有Fast Open选项且cookie域为空的SYN包
  \item 服务器生成一个cookie,通过SYN-ACK包回复给客户端
  \item 客户端将该cookie缓存,并用于后续的TCP Fast Open连接。
\end{enumerate}

建立Fast Open连接的过程如下:
\begin{enumerate}
  \item 客户端发送一个带有数据的SYN包,同时在Fast Open选项里带上cookie。
  \item 服务端验证cookie的有效性,如果有效,则回复SYN-ACK,并将数据发给应用程序。
  \item 客户端发送ACK请求,确认服务器发来的SYN-ACK包及其中的数据。
  \item 至此,三次握手已完成,后续过程与普通TCP连接一致。
\end{enumerate}

\subsubsection{Fast Open选项格式}
\label{subsubsec:fast-open-option}
该选项的格式如下:
\begin{minted}[linenos]{text}
                                   +-+-+-+-+-+-+-+-+-+-+-+-+-+-+-+-+
                                   |  Kind = 34    |    Length     |
   +-+-+-+-+-+-+-+-+-+-+-+-+-+-+-+-+-+-+-+-+-+-+-+-+-+-+-+-+-+-+-+-+
   |                                                               |
   ~                            Cookie                             ~
   |                                                               |
   +-+-+-+-+-+-+-+-+-+-+-+-+-+-+-+-+-+-+-+-+-+-+-+-+-+-+-+-+-+-+-+-+
   Length          1 byte: range 6 to 18 (bytes); limited by
                           remaining space in the options field.
                           The number MUST be even.
   Cookie          0, or 4 to 16 bytes (Length - 2)
\end{minted}
这里注意,虽然图示中Cookie按照32位对齐了,但是这并不是必须的。
