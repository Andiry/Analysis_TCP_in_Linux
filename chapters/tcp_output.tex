\chapter{TCP输出}
\label{chapter:tcp_output}

\minitoc

\section{\mintinline{c}{tcp_sendmsg}}
使用TCP发送数据的大部分工作都是在\mintinline{c}{tcp_sendmsg}函数中实现的。

\begin{minted}[linenos]{c}
/* Location: net/ipv4/tcp.c
 *
 * Parameter:
 *        sk 传输所使用的套接字
 *        msg 要传输的用户层的数据包
 *        size 用户要传输的数据的大小
 */
int tcp_sendmsg(struct sock *sk, struct msghdr *msg, size_t size)
{
        struct tcp_sock *tp = tcp_sk(sk);
        struct sk_buff *skb;
        int flags, err, copied = 0;
        int mss_now = 0, size_goal, copied_syn = 0;
        bool sg;
        long timeo;

        /* 对套接字加锁 */
        lock_sock(sk);

        flags = msg->msg_flags;
        if (flags & MSG_FASTOPEN) {
                err = tcp_sendmsg_fastopen(sk, msg, &copied_syn, size);
                if (err == -EINPROGRESS && copied_syn > 0)
                        goto out;
                else if (err)
                        goto out_err;
        }
\end{minted}
根据调用者传入的标志位,判断是否启用了快速开启(Fast Open)。关于Fast Open的讨论,详见
\ref{subsec:rfc7413}。

\begin{minted}[linenos]{c}
        timeo = sock_sndtimeo(sk, flags & MSG_DONTWAIT);

        /* Wait for a connection to finish. One exception is TCP Fast Open
         * (passive side) where data is allowed to be sent before a connection
         * is fully established.
         */
        if (((1 << sk->sk_state) & ~(TCPF_ESTABLISHED | TCPF_CLOSE_WAIT)) &&
            !tcp_passive_fastopen(sk)) {
                err = sk_stream_wait_connect(sk, &timeo);
                if (err != 0)
                        goto do_error;
        }

        if (unlikely(tp->repair)) {
                if (tp->repair_queue == TCP_RECV_QUEUE) {
                        copied = tcp_send_rcvq(sk, msg, size);
                        goto out_nopush;
                }

                err = -EINVAL;
                if (tp->repair_queue == TCP_NO_QUEUE)
                        goto out_err;

                /* 'common' sending to sendq */
        }

        /* This should be in poll */
        sk_clear_bit(SOCKWQ_ASYNC_NOSPACE, sk);

        mss_now = tcp_send_mss(sk, &size_goal, flags);

        /* Ok commence sending. */
        copied = 0;

        err = -EPIPE;
        if (sk->sk_err || (sk->sk_shutdown & SEND_SHUTDOWN))
                goto out_err;

        sg = !!(sk->sk_route_caps & NETIF_F_SG);

        while (msg_data_left(msg)) {
                int copy = 0;
                int max = size_goal;

                skb = tcp_write_queue_tail(sk);
                if (tcp_send_head(sk)) {
                        if (skb->ip_summed == CHECKSUM_NONE)
                                max = mss_now;
                        copy = max - skb->len;
                }

                if (copy <= 0) {
new_segment:
                        /* Allocate new segment. If the interface is SG,
                         * allocate skb fitting to single page.
                         */
                        if (!sk_stream_memory_free(sk))
                                goto wait_for_sndbuf;

                        skb = sk_stream_alloc_skb(sk,
                                                  select_size(sk, sg),
                                                  sk->sk_allocation,
                                                  skb_queue_empty(&sk->sk_write_queue));
                        if (!skb)
                                goto wait_for_memory;

                        /*
                         * Check whether we can use HW checksum.
                         */
                        if (sk->sk_route_caps & NETIF_F_ALL_CSUM)
                                skb->ip_summed = CHECKSUM_PARTIAL;

                        skb_entail(sk, skb);
                        copy = size_goal;
                        max = size_goal;

                        /* All packets are restored as if they have
                         * already been sent. skb_mstamp isn't set to
                         * avoid wrong rtt estimation.
                         */
                        if (tp->repair)
                                TCP_SKB_CB(skb)->sacked |= TCPCB_REPAIRED;
                }

                /* Try to append data to the end of skb. */
                if (copy > msg_data_left(msg))
                        copy = msg_data_left(msg);

                /* Where to copy to? */
                if (skb_availroom(skb) > 0) {
                        /* We have some space in skb head. Superb! */
                        copy = min_t(int, copy, skb_availroom(skb));
                        err = skb_add_data_nocache(sk, skb, &msg->msg_iter, copy);
                        if (err)
                                goto do_fault;
                } else {
                        bool merge = true;
                        int i = skb_shinfo(skb)->nr_frags;
                        struct page_frag *pfrag = sk_page_frag(sk);

                        if (!sk_page_frag_refill(sk, pfrag))
                                goto wait_for_memory;

                        if (!skb_can_coalesce(skb, i, pfrag->page,
                                              pfrag->offset)) {
                                if (i == sysctl_max_skb_frags || !sg) {
                                        tcp_mark_push(tp, skb);
                                        goto new_segment;
                                }
                                merge = false;
                        }

                        copy = min_t(int, copy, pfrag->size - pfrag->offset);

                        if (!sk_wmem_schedule(sk, copy))
                                goto wait_for_memory;

                        err = skb_copy_to_page_nocache(sk, &msg->msg_iter, skb,
                                                       pfrag->page,
                                                       pfrag->offset,
                                                       copy);
                        if (err)
                                goto do_error;

                        /* Update the skb. */
                        if (merge) {
                                skb_frag_size_add(&skb_shinfo(skb)->frags[i - 1], copy);
                        } else {
                                skb_fill_page_desc(skb, i, pfrag->page,
                                                   pfrag->offset, copy);
                                get_page(pfrag->page);
                        }
                        pfrag->offset += copy;
                }

                if (!copied)
                        TCP_SKB_CB(skb)->tcp_flags &= ~TCPHDR_PSH;

                tp->write_seq += copy;
                TCP_SKB_CB(skb)->end_seq += copy;
                tcp_skb_pcount_set(skb, 0);

                copied += copy;
                if (!msg_data_left(msg)) {
                        tcp_tx_timestamp(sk, skb);
                        goto out;
                }

                if (skb->len < max || (flags & MSG_OOB) || unlikely(tp->repair))
                        continue;

                if (forced_push(tp)) {
                        tcp_mark_push(tp, skb);
                        __tcp_push_pending_frames(sk, mss_now, TCP_NAGLE_PUSH);
                } else if (skb == tcp_send_head(sk))
                        tcp_push_one(sk, mss_now);
                continue;

wait_for_sndbuf:
                set_bit(SOCK_NOSPACE, &sk->sk_socket->flags);
wait_for_memory:
                if (copied)
                        tcp_push(sk, flags & ~MSG_MORE, mss_now,
                                 TCP_NAGLE_PUSH, size_goal);

                err = sk_stream_wait_memory(sk, &timeo);
                if (err != 0)
                        goto do_error;

                mss_now = tcp_send_mss(sk, &size_goal, flags);
        }

out:
        if (copied)
                tcp_push(sk, flags, mss_now, tp->nonagle, size_goal);
out_nopush:
        release_sock(sk);
        return copied + copied_syn;

do_fault:
        if (!skb->len) {
                tcp_unlink_write_queue(skb, sk);
                /* It is the one place in all of TCP, except connection
                 * reset, where we can be unlinking the send_head.
                 */
                tcp_check_send_head(sk, skb);
                sk_wmem_free_skb(sk, skb);
        }

do_error:
        if (copied + copied_syn)
                goto out;
out_err:
        err = sk_stream_error(sk, flags, err);
        /* make sure we wake any epoll edge trigger waiter */
        if (unlikely(skb_queue_len(&sk->sk_write_queue) == 0 && err == -EAGAIN))
                sk->sk_write_space(sk);
        release_sock(sk);
        return err;
}
\end{minted}

\section{tcp\_transmit\_skb}

\begin{minted}[linenos]{c}
/* 为SYN包计算TCP选项,这个函数中计算出来的还不是最终的格式。
 */
static unsigned int tcp_syn_options(struct sock *sk, struct sk_buff *skb,
                                struct tcp_out_options *opts,
                                struct tcp_md5sig_key **md5);
/* 为已经建立连接的套接字计算TCP选项,这个函数中计算出来的还不是最终的格式。
 */
static unsigned int tcp_established_options(struct sock *sk, struct sk_buff *skb,
                                        struct tcp_out_options *opts,
                                        struct tcp_md5sig_key **md5);
/* 在skb中为头部流出空间。
 */
skb_push(skb, tcp_header_size);
/* 判断skb是否为一个纯ACK。这里把实现也放出来了。可以看到,纯ACK的包最显著
 * 的特点是其长度。Linux里通过判断长度直接快速判断出skb是否为一个纯ACK。
 */
static inline bool skb_is_tcp_pure_ack(const struct sk_buff *skb)
{
        return skb->truesize == 2;
}
/* 重置传输层的header的指针?
 */
static inline void skb_reset_transport_header(struct sk_buff *skb)
{
        skb->transport_header = skb->data - skb->head;
}
/* 选择发送窗口的大小
 */
tcp_select_window(sk);
tcp_urg_mode(tp);
before(tcb->seq, tp->snd_up);
tcp_options_write((__be32 *)(th + 1), tp, &opts);
tcp_event_ack_sent(sk, tcp_skb_pcount(skb));
tcp_event_data_sent(tp, sk);
queue_xmit();
tcp_enter_cwr(sk);
net_xmit_eval(err);
\end{minted}

\section{tcp\_select\_window(struct sk\_buff *skb)}
这个函数的作用是选择一个新的窗口大小以用于更新\mintinline{c}{tcp_sock}。
返回的结果根据RFC1323(详见\ref{subsec:rfc1323})进行了缩放。

\subsection{代码分析}
\begin{minted}[linenos]{c}
static u16 tcp_select_window(struct sock *sk)
{
        struct tcp_sock *tp = tcp_sk(sk);
        u32 old_win = tp->rcv_wnd;
        u32 cur_win = tcp_receive_window(tp);
        u32 new_win = __tcp_select_window(sk);
        /* old_win是接收方窗口的大小。
         * cur_win当前的接收窗口大小。
         * new_win是新选择出来的窗口大小。
         */

        /* 当新窗口的大小小于当前窗口的大小时,不能缩减窗口大小。
         * 这是IEEE强烈不建议的一种行为。
         */
        if (new_win < cur_win) {
                /* Danger Will Robinson!
                 * Don't update rcv_wup/rcv_wnd here or else
                 * we will not be able to advertise a zero
                 * window in time.  --DaveM
                 *
                 * Relax Will Robinson.
                 */
                if (new_win == 0)
                        NET_INC_STATS(sock_net(sk),
                                      LINUX_MIB_TCPWANTZEROWINDOWADV);
                /* 当计算出来的新窗口小于当前窗口时,将新窗口设置为大于cur_win
                 * 的1<<tp->rx_opt.rcv_wscale的整数倍。
                 */
                new_win = ALIGN(cur_win, 1 << tp->rx_opt.rcv_wscale);
        }
        /* 将当前的接收窗口设置为新的窗口大小。*/
        tp->rcv_wnd = new_win;
        tp->rcv_wup = tp->rcv_nxt;

        /* 判断当前窗口未越界。*/
        if (!tp->rx_opt.rcv_wscale && sysctl_tcp_workaround_signed_windows)
                new_win = min(new_win, MAX_TCP_WINDOW);
        else
                new_win = min(new_win, (65535U << tp->rx_opt.rcv_wscale));

        /* RFC1323 缩放窗口大小。这里之所以是右移,是因为此时的new_win是
         * 窗口的真正大小。所以返回时需要返回正常的可以放在16位整型中的窗口大小。
         * 所以需要右移。
         */
        new_win >>= tp->rx_opt.rcv_wscale;

        /* If we advertise zero window, disable fast path. */
        if (new_win == 0) {
                tp->pred_flags = 0;
                if (old_win)
                        NET_INC_STATS(sock_net(sk),
                                      LINUX_MIB_TCPTOZEROWINDOWADV);
        } else if (old_win == 0) {
                NET_INC_STATS(sock_net(sk), LINUX_MIB_TCPFROMZEROWINDOWADV);
        }

        return new_win;
}
\end{minted}

在这个过程中,还调用了\mintinline{c}{__tcp_select_window(sk)}来计算新的窗口大小。
该函数会尝试增加窗口的大小,但是有两个限制条件:

\begin{enumerate}
  \item 窗口不能收缩(RFC793)
  \item 每个socket所能使用的内存是有限制的。
\end{enumerate}

RFC 1122中说:
\begin{quote}
"the suggested [SWS] avoidance algorithm for the receiver is to keep
RECV.NEXT + RCV.WIN fixed until:
RCV.BUFF - RCV.USER - RCV.WINDOW >= min(1/2 RCV.BUFF, MSS)"

推荐的用于接收方的糊涂窗口综合症的避免算法是保持recv.next+rcv.win不变,直到:
RCV.BUFF - RCV.USER - RCV.WINDOW >= min(1/2 RCV.BUFF, MSS)
\end{quote}

换句话说,就是除非缓存的大小多出来至少一个MSS那么多字节,否则不要增长窗口右边界
的大小。

然而,根据Linux注释中的说法,被推荐的这个算法会破坏头预测(header prediction),
因为头预测会假定\mintinline{c}{th->window}不变。严格地说,
保持\mintinline{c}{th->window}固定不变会违背接收方的用于防止糊涂窗口综合症的准则。
在这种规则下,一个单字节的包的流会引发窗口的右边界总是提前一个字节。
当然,如果发送方实现了预防糊涂窗口综合症的方法,那么就不会出现问题。

Linux的TCP部分的作者们参考了BSD的实现方法。BSD在这方面的做法是是,
如果空闲空间小于最大可用空间的$\frac{1}{4}$,且空闲空间
小于mss的$\frac{1}{2}$,那么就把窗口设置为0。否则,只是单纯地阻止窗口缩小,
或者阻止窗口大于最大可表示的范围(the largest representable value)。
BSD的方法似乎“意外地”使得窗口基本上都是MSS的整倍数。且很多情况下窗口大小都是
固定不变的。因此,Linux采用强制窗口为MSS的整倍数,以获得相似的行为。

\begin{minted}[linenos]{c}
u32 __tcp_select_window(struct sock *sk)
{
        struct inet_connection_sock *icsk = inet_csk(sk);
        struct tcp_sock *tp = tcp_sk(sk);
        int mss = icsk->icsk_ack.rcv_mss;
        int free_space = tcp_space(sk);
        int allowed_space = tcp_full_space(sk);
        int full_space = min_t(int, tp->window_clamp, allowed_space);
        int window;

        /* 如果mss超过了总共的空间大小,那么把mss限制在允许的空间范围内。 */
        if (mss > full_space)
                mss = full_space;

        if (free_space < (full_space >> 1)) {
                /* 当空闲空间小于允许空间的一半时。 */
                icsk->icsk_ack.quick = 0;

                if (tcp_under_memory_pressure(sk))
                        tp->rcv_ssthresh = min(tp->rcv_ssthresh,
                                               4U * tp->advmss);

                /* free_space有可能成为新的窗口的大小,因此,需要考虑
                 * 窗口扩展的影响。
                 */
                free_space = round_down(free_space, 1 << tp->rx_opt.rcv_wscale);

                /* 如果空闲空间小于mss的大小,或者低于最大允许空间的的1/16,那么,
                 * 返回0窗口。否则,tcp_clamp_window()会增长接收缓存到tcp_rmem[2]。
                 * 新进入的数据会由于内醋限制而被丢弃。对于较大的窗口,单纯地探测mss的
                 * 大小以宣告0窗口有些太晚了(可能会超过限制)。
                 */
                if (free_space < (allowed_space >> 4) || free_space < mss)
                        return 0;
        }

        if (free_space > tp->rcv_ssthresh)
                free_space = tp->rcv_ssthresh;

        /* 这里处理一个例外情况,就是如果开启了窗口缩放,那么就没法对齐mss了。
         * 所以就保持窗口是对齐2的幂的。
         */
        window = tp->rcv_wnd;
        if (tp->rx_opt.rcv_wscale) {
                window = free_space;

                /* Advertise enough space so that it won't get scaled away.
                 * Import case: prevent zero window announcement if
                 * 1<<rcv_wscale > mss.
                 */
                if (((window >> tp->rx_opt.rcv_wscale) << tp->rx_opt.rcv_wscale) != window)
                        window = (((window >> tp->rx_opt.rcv_wscale) + 1)
                                  << tp->rx_opt.rcv_wscale);
        } else {
                /* 如果内存条件允许,那么就把窗口设置为mss的整倍数。
                 * 或者如果free_space > 当前窗口大小加上全部允许的空间的一半,
                 * 那么,就将窗口大小设置为free_space
                 */
                if (window <= free_space - mss || window > free_space)
                        window = (free_space / mss) * mss;
                else if (mss == full_space &&
                         free_space > window + (full_space >> 1))
                        window = free_space;
        }

        return window;
}
\end{minted}
 
 
