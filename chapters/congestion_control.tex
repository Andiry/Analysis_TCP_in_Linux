\section{拥塞控制实现}

	\subsection{拥塞控制状态机}
		\subsubsection{基本转换}
\begin{minted}[linenos]{C}
enum tcp_ca_state {
	TCP_CA_Open = 0,
#define TCPF_CA_Open	(1<<TCP_CA_Open)
	TCP_CA_Disorder = 1,
#define TCPF_CA_Disorder (1<<TCP_CA_Disorder)
	TCP_CA_CWR = 2,
#define TCPF_CA_CWR	(1<<TCP_CA_CWR)
	TCP_CA_Recovery = 3,
#define TCPF_CA_Recovery (1<<TCP_CA_Recovery)
	TCP_CA_Loss = 4
#define TCPF_CA_Loss	(1<<TCP_CA_Loss)
};
\end{minted}
		\subsubsection{Open状态}
			Open状态是常态,在这种状态下TCP发送方通过优化后的快速路径来处理接收ACK。当一个确认到达时,发送方根据拥塞窗口时小于还是大于慢启动阙值,按慢启动或者拥塞避免来增大拥塞窗口。

		\subsubsection{Disorder状态}
			当发送方检测到DACK(重复确认)或者SACK(选择性确认)时,将转变为Disorder(无序)状态。在该状态下,拥塞窗口不做调整,而是每个新到的段出发一个新的数据段的发送。因此,TCP发送方遵循包守恒原则,该原则规定一个新包只有在一个老的包离开网络后才发送。在实践中该规定的表现类似于IETF的传输提议,允许当拥塞窗口较小或是上个传输窗口中有大量数据段丢失时,使用快速重传以更有效地恢复。

		\subsubsection{CWR状态}
			TCP发送方可能从显示拥塞通知、ICMP源端抑制(ICMP source quench)或是本地设备接收到拥塞通知。当收到一个拥塞通知时,发送方并不立刻减小拥塞窗口,而是每隔一个新到的ACK减小一个段直到窗口的大小减半为止。发送方在减小拥塞窗口大小的过程中不会有明显的重传,这就处于CWR(Congestion Window Reduced,拥塞窗口减小)状态。CWR状态可以被Revcovery状态或者Loss状态中断。进入拥塞窗口减小的函数如下:

\begin{minted}[linenos]{C}
/* 
Location:

	net/ipv4/tcp_input.c

Function:

	Enter CWR state. Disable cwnd undo since congestion is proven with ECN 

Parameter:

	sk:传输控制块
*/
void tcp_enter_cwr(struct sock *sk)
{
	struct tcp_sock *tp = tcp_sk(sk);
	/*进入CWR后就不需要窗口撤消了,
      因此需要清除拥塞控制的慢启动阙值
	*/
	tp->prior_ssthresh = 0;
	/*可以看出只有OPen状态和Disorder状态可以转移到该状态*/
	if (inet_csk(sk)->icsk_ca_state < TCP_CA_CWR) {
		/*湖北人与CWR状态后不允许在进行拥塞窗口撤消了*/		
		tp->undo_marker = 0;
		/*进行相关的初始化*/
		tcp_init_cwnd_reduction(sk);
		/*设置状态*/		
		tcp_set_ca_state(sk, TCP_CA_CWR);
	}
}
\end{minted}

		对于函数\mintinline{C}{tcp_init_cwnd_reduction}的调用如下:
\begin{minted}[linenos]{C}
/*
Location:

	net/ipv4/tcp_input.c

Function:
	The cwnd reduction in CWR and Recovery uses the PRR algorithm in RFC 6937.
	It computes the number of packets to send (sndcnt) based on packets newly
	delivered:
		1) If the packets in flight is larger than ssthresh, PRR spreads the
			cwnd reductions across a full RTT.
		2) Otherwise PRR uses packet conservation to send as much as delivered.
			But when the retransmits are acked without further losses, PRR
			slow starts cwnd up to ssthresh to speed up the recovery.

parameter:

	sk:传输控制块
*/
static void tcp_init_cwnd_reduction(struct sock *sk)
{
	struct tcp_sock *tp = tcp_sk(sk);

	/*记录发送拥塞时的snd.nxt*/
	tp->high_seq = tp->snd_nxt;
	/*snd_nxt at the time of TLP retransmit*/
	tp->tlp_high_seq = 0;
	/*snd_cwnd_cnt表示自从上次调整拥塞窗口到
	 目前为止接收到的总ACK段数,这里设置为0是因为
	 刚刚改变拥塞控制算法为PRR。
	*/
	tp->snd_cwnd_cnt = 0;
	/*prior_cwnd means Congestion window at start of Recovery
		设置该值为当前拥塞窗口。
	*/
	tp->prior_cwnd = tp->snd_cwnd;
	/*Number of newly delivered packets
	  to receiver in Recovery,设置为0
	*/
	tp->prr_delivered = 0;
	/*Total number of pkts sent during Recovery*/
	tp->prr_out = 0;
	/*根据给定的拥塞控制算法重新设置拥塞慢启动阙值*/
	tp->snd_ssthresh = inet_csk(sk)->icsk_ca_ops->ssthresh(sk);
	/*设置TCP_ECN_QUEUE_CWR标志,标识由于收到显示拥塞通知而进入拥塞状态*/
	tcp_ecn_queue_cwr(tp);
}
\end{minted}
	\subsection{拥塞控制状态的处理及转换}

	\subsection{显式拥塞通知(ECN)}


