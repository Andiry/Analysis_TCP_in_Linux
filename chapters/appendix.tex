\chapter{附录:基础知识}

\minitoc
	\section{计算机底层知识}
		\subsection{机器数}
			在计算机内的数(称之为“机器数”)值有3种表示法:原码、反码和补码。
			to do ???? 添加关于反码运算的规则,校验码的计算.
			\subsubsection{正负数与定点数浮点数的表示}
				在计算机内,通常把二进制数的最高位定义为符号位,用0表示正数,用
				1表示负数。其余表示数值。

				规定小数点位置固定不变的数称为定点数,小数点位置可以变的数称为浮点数。
				
			\subsubsection{原码}
				原码(true form)是一种计算机中对数字的二进制定点表示方法。
				原码表示法在数值前面增加了一位符号位(即最高位为符号位):
				正数该位为0,负数该位为1(0有两种表示:+0和-0),
				其余位表示数值的大小。
			\subsubsection{反码}
				正数的反码与其原码相同;负数的反码是对其原码逐位取反,但符号位除外。
				
				反码的运算规则。

			\subsubsection{补码}
				正数的补码就是其本身,负数的补码是在其原码的基础上, 符号位不变, 其余各位取反, 最后+1。
				
				由于补码利于运算,故而计算机中数都是利用补码表示的。
	\section{GNU/LINUX}
		\subsection{错误处理}
			\subsubsection{错误码}
\begin{minted}[linenos]{C}
/*
Location:

	1 -  34   include/uapi/asm-generic/errno-base.h
	35- 133	  include/uapi/asm-generic/errno.h

Function:

	定义了Linux中的错误码。
*/
#define	EPERM		 1	/* Operation not permitted */
#define	ENOENT		 2	/* No such file or directory */
#define	ESRCH		 3	/* No such process */
#define	EINTR		 4	/* Interrupted system call */
#define	EIO		 	 5	/* I/O error */
#define	ENXIO		 6	/* No such device or address */
#define	E2BIG		 7	/* Argument list too long */
#define	ENOEXEC		 8	/* Exec format error */
#define	EBADF		 9	/* Bad file number */
#define	ECHILD		10	/* No child processes */
#define	EAGAIN		11	/* Try again */
#define	ENOMEM		12	/* Out of memory */
#define	EACCES		13	/* Permission denied */
#define	EFAULT		14	/* Bad address */
#define	ENOTBLK		15	/* Block device required */
#define	EBUSY		16	/* Device or resource busy */
#define	EEXIST		17	/* File exists */
#define	EXDEV		18	/* Cross-device link */
#define	ENODEV		19	/* No such device */
#define	ENOTDIR		20	/* Not a directory */
#define	EISDIR		21	/* Is a directory */
#define	EINVAL		22	/* Invalid argument */
#define	ENFILE		23	/* File table overflow */
#define	EMFILE		24	/* Too many open files */
#define	ENOTTY		25	/* Not a typewriter */
#define	ETXTBSY		26	/* Text file busy */
#define	EFBIG		27	/* File too large */
#define	ENOSPC		28	/* No space left on device */
#define	ESPIPE		29	/* Illegal seek */
#define	EROFS		30	/* Read-only file system */
#define	EMLINK		31	/* Too many links */
#define	EPIPE		32	/* Broken pipe */
#define	EDOM		33	/* Math argument out of domain of func */
#define	ERANGE		34	/* Math result not representable */

#define	EDEADLK		35	/* Resource deadlock would occur */
#define	ENAMETOOLONG	36	/* File name too long */
#define	ENOLCK		37	/* No record locks available */

/*
 * This error code is special: arch syscall entry code will return
 * -ENOSYS if users try to call a syscall that doesn't exist.  To keep
 * failures of syscalls that really do exist distinguishable from
 * failures due to attempts to use a nonexistent syscall, syscall
 * implementations should refrain from returning -ENOSYS.
 */
#define	ENOSYS		38	/* Invalid system call number */

#define	ENOTEMPTY	39	/* Directory not empty */
#define	ELOOP		40	/* Too many symbolic links encountered */
#define	EWOULDBLOCK	EAGAIN	/* Operation would block */
#define	ENOMSG		42	/* No message of desired type */
#define	EIDRM		43	/* Identifier removed */
#define	ECHRNG		44	/* Channel number out of range */
#define	EL2NSYNC	45	/* Level 2 not synchronized */
#define	EL3HLT		46	/* Level 3 halted */
#define	EL3RST		47	/* Level 3 reset */
#define	ELNRNG		48	/* Link number out of range */
#define	EUNATCH		49	/* Protocol driver not attached */
#define	ENOCSI		50	/* No CSI structure available */
#define	EL2HLT		51	/* Level 2 halted */
#define	EBADE		52	/* Invalid exchange */
#define	EBADR		53	/* Invalid request descriptor */
#define	EXFULL		54	/* Exchange full */
#define	ENOANO		55	/* No anode */
#define	EBADRQC		56	/* Invalid request code */
#define	EBADSLT		57	/* Invalid slot */

#define	EDEADLOCK	EDEADLK

#define	EBFONT		59	/* Bad font file format */
#define	ENOSTR		60	/* Device not a stream */
#define	ENODATA		61	/* No data available */
#define	ETIME		62	/* Timer expired */
#define	ENOSR		63	/* Out of streams resources */
#define	ENONET		64	/* Machine is not on the network */
#define	ENOPKG		65	/* Package not installed */
#define	EREMOTE		66	/* Object is remote */
#define	ENOLINK		67	/* Link has been severed */
#define	EADV		68	/* Advertise error */
#define	ESRMNT		69	/* Srmount error */
#define	ECOMM		70	/* Communication error on send */
#define	EPROTO		71	/* Protocol error */
#define	EMULTIHOP	72	/* Multihop attempted */
#define	EDOTDOT		73	/* RFS specific error */
#define	EBADMSG		74	/* Not a data message */
#define	EOVERFLOW	75	/* Value too large for defined data type */
#define	ENOTUNIQ	76	/* Name not unique on network */
#define	EBADFD		77	/* File descriptor in bad state */
#define	EREMCHG		78	/* Remote address changed */
#define	ELIBACC		79	/* Can not access a needed shared library */
#define	ELIBBAD		80	/* Accessing a corrupted shared library */
#define	ELIBSCN		81	/* .lib section in a.out corrupted */
#define	ELIBMAX		82	/* Attempting to link in too many shared libraries */
#define	ELIBEXEC	83	/* Cannot exec a shared library directly */
#define	EILSEQ		84	/* Illegal byte sequence */
#define	ERESTART	85	/* Interrupted system call should be restarted */
#define	ESTRPIPE	86	/* Streams pipe error */
#define	EUSERS		87	/* Too many users */
#define	ENOTSOCK	88	/* Socket operation on non-socket */
#define	EDESTADDRREQ	89	/* Destination address required */
#define	EMSGSIZE	90	/* Message too long */
#define	EPROTOTYPE	91	/* Protocol wrong type for socket */
#define	ENOPROTOOPT	92	/* Protocol not available */
#define	EPROTONOSUPPORT	93	/* Protocol not supported */
#define	ESOCKTNOSUPPORT	94	/* Socket type not supported */
#define	EOPNOTSUPP	95	/* Operation not supported on transport endpoint */
#define	EPFNOSUPPORT	96	/* Protocol family not supported */
#define	EAFNOSUPPORT	97	/* Address family not supported by protocol */
#define	EADDRINUSE	98	/* Address already in use */
#define	EADDRNOTAVAIL	99	/* Cannot assign requested address */
#define	ENETDOWN	100	/* Network is down */
#define	ENETUNREACH	101	/* Network is unreachable */
#define	ENETRESET	102	/* Network dropped connection because of reset */
#define	ECONNABORTED	103	/* Software caused connection abort */
#define	ECONNRESET	104	/* Connection reset by peer */
#define	ENOBUFS		105	/* No buffer space available */
#define	EISCONN		106	/* Transport endpoint is already connected */
#define	ENOTCONN	107	/* Transport endpoint is not connected */
#define	ESHUTDOWN	108	/* Cannot send after transport endpoint shutdown */
#define	ETOOMANYREFS	109	/* Too many references: cannot splice */
#define	ETIMEDOUT	110	/* Connection timed out */
#define	ECONNREFUSED	111	/* Connection refused */
#define	EHOSTDOWN	112	/* Host is down */
#define	EHOSTUNREACH	113	/* No route to host */
#define	EALREADY	114	/* Operation already in progress */
#define	EINPROGRESS	115	/* Operation now in progress */
#define	ESTALE		116	/* Stale file handle */
#define	EUCLEAN		117	/* Structure needs cleaning */
#define	ENOTNAM		118	/* Not a XENIX named type file */
#define	ENAVAIL		119	/* No XENIX semaphores available */
#define	EISNAM		120	/* Is a named type file */
#define	EREMOTEIO	121	/* Remote I/O error */
#define	EDQUOT		122	/* Quota exceeded */

#define	ENOMEDIUM	123	/* No medium found */
#define	EMEDIUMTYPE	124	/* Wrong medium type */
#define	ECANCELED	125	/* Operation Canceled */
#define	ENOKEY		126	/* Required key not available */
#define	EKEYEXPIRED	127	/* Key has expired */
#define	EKEYREVOKED	128	/* Key has been revoked */
#define	EKEYREJECTED	129	/* Key was rejected by service */

/* for robust mutexes */
#define	EOWNERDEAD	130	/* Owner died */
#define	ENOTRECOVERABLE	131	/* State not recoverable */

#define ERFKILL		132	/* Operation not possible due to RF-kill */

#define EHWPOISON	133	/* Memory page has hardware error */
\end{minted}
			\subsubsection{\mintinline{C}{IS_ERR,PTR_ERR,ERR_PTR}}

				相关定义如下:
\begin{minted}[linenos]{C}
/*
Location:

	include/linux/err.h

Description:
 	Kernel pointers have redundant information, so we can use a
	scheme where we can return either an error code or a normal
	pointer with the same return value.

	This should be a per-architecture thing, to allow different
	error and pointer decisions.

*/
#define MAX_ERRNO	4095

#ifndef __ASSEMBLY__

#define IS_ERR_VALUE(x) unlikely((x) >= (unsigned long)-MAX_ERRNO)

static inline void * __must_check ERR_PTR(long error)
{
	return (void *) error;
}

static inline long __must_check PTR_ERR(__force const void *ptr)
{
	return (long) ptr;
}

static inline bool __must_check IS_ERR(__force const void *ptr)
{
	return IS_ERR_VALUE((unsigned long)ptr);
}

static inline bool __must_check IS_ERR_OR_NULL(__force const void *ptr)
{
	return !ptr || IS_ERR_VALUE((unsigned long)ptr);
}
\end{minted}


				要想明白上述的代码,首先要理解内核空间。所有的驱动程序都是运行在内核空间,内核空间虽然很大,但总是有限的,而在这有限的空间中,其最后一个page是专门保留的,也就是说一般人不可能用到内核空间最后一个page的指针。换句话说,你在写设备驱动程序的过程中,涉及到的任何一个指针,必然有三种情况:有效指针;NULL,空指针;错误指针,或者说无效指针。

				而所谓的错误指针就是指其已经到达了最后一个page,即内核用最后一页捕捉错误。比如对于32bit的系统来说,内核空间最高地址0xffffffff,那么最后一个page就是指的0xfffff000~0xffffffff(假设4k一个page),这段地址是被保留的。内核空间为什么留出最后一个page?我们知道一个page可能是4k,也可能是更多,比如8k,但至少它也是4k,留出一个page出来就可以让我们把内核空间的指针来记录错误了。内核返回的指针一般是指向页面的边界(4k边界),即\mintinline{C}{ptr & 0xfff == 0}。如果你发现你的一个指针指向这个范围中的某个地址,那么你的代码肯定出错了。IS\_ERR()就是判断指针是否有错,如果指针并不是指向最后一个page,那么没有问题;如果指针指向了最后一个page,那么说明实际上这不是一个有效的指针,这个指针里保存的实际上是一种错误代码。而通常很常用的方法就是先用IS\_ERR()来判断是否是错误,然后如果是,那么就调用PTR\_ERR()来返回这个错误代码。因此,判断一个指针是不是有效的,我们可以调用宏IS\_ERR\_VALUE,即判断指针是不是在(0xfffff000,0xffffffff)之间,因此,可以用IS\_ERR()来判断内核函数的返回值是不是一个有效的指针。注意这里用unlikely()的用意!

				至于PTR\_ERR(), ERR\_PTR(),只是强制转换以下而已。而PTR\_ERR()只是返回错误代码,也就是提供一个信息给调用者,如果你只需要知道是否出错,而不在乎因为什么而出错,那你当然不用调用PTR\_ERR()了。

				如果指针指向了最后一个page,那么说明实际上这不是一个有效的指针。这个指针里保存的实际上是一种错误代码。而通常很常用的方法就是先用IS\_ERR()来判断是否是错误,然后如果是,那么就调用PTR\_ERR()来返回这个错误代码。
		\subsection{调试函数}
			\subsubsection{BUG\_ON}
\begin{minted}[linenos]{C}
   #define BUG_ON(condition) do { /
         if (unlikely((condition)!=0)) /
         BUG(); /
    } while(0)
\end{minted}
    			如果在程序的执行过程中,觉得该condition下是一个BUG,
				可以添加此调试信息,查看对应堆栈内容。
			\subsubsection{WARN\_ON}			
    			而WARN\_ON则是调用dump\_stack,打印堆栈信息,不会OOPS
\begin{minted}[linenos]{C}
  #define WARN_ON(condition) do { /
      if (unlikely((condition)!=0)) { /
             printk("Badness in %s at %s:%d/n", __FUNCTION__, __FILE__, __LINE__); /
             dump_stack(); /
        } /
     } while (0)
\end{minted}

	\section{C语言}
		\subsection{结构体初始化}
\begin{minted}[linenos]{C}
typedef struct{
	int  a;
	char ch;
}flag;
/*
	目的是将a初始化为1,ch初始化为'u'.
*/
/*法一:分别初始化*/
flag tmp;
tmp.a=1;
tmp.ch='u';

/*法二:点式初始化*/
flag tmp={.a=1,.ch='u'};		//注意两个变量之间使用 , 而不是;
/*法三:*/
flag tmp={
			a:1,
			ch:'u'
		};
\end{minted}
			当然,我们也可以使用上述任何一种方法只对结构体中的某几个变量进行初始化。

		\subsection{位字段}
			在存储空间极为宝贵的情况下,有可能需要将多个对象保存在一个机器字中。而在linux开发的早期,那时确实空间极其宝贵。于是乎,那一帮黑客们就发明了各种各样的办法。一种常用的办法是使用类似于编译器符号表的单个二进制位标志集合,即定义一系列的2的指数次方的数,此方法确实有效。但是,仍然十分浪费空间。而且有可能很多位都利用不到。于是乎,他们提出了另一种新的思路即位字段。我们可以利用如下方式定义一个包含3位的变量。

\begin{minted}[linenos]{C}
struct {
	unsigned int a:1;
	unsigned int b:1;
	unsigned int c:1;
}flags;
\end{minted}
			
			字段可以不命名,无名字段,即只有一个冒号和宽度,起到填充作用。特殊宽度0可以用来强制在下一个字边界上对齐,一般位于结构体的尾部。
			
			冒号后面表示相应字段的宽度(二进制宽度),即不一定非得是1位。字段被声明为\mintinline{C}{unsigned int}类型,以确保它们是无符号量。

			当然我们需要注意,机器是分大端和小端存储的。因此,我们在选择外部定义数据的情况下爱,必须仔细考虑那一端优先的问题。同时,字段不是数组,并且没有地址,因此不能对它们使用\mintinline{C}{&}运算符。	
	\section{GCC}
		\subsection{\mintinline{C}{__attribute__}}
			GNU C的一大特色就是\mintinline{C}{__attribute__}机制。\mintinline{C}{__attribute__}可以设置函数属性(Function Attribute)、变量属性(Variable Attribute)和类型属性(Type Attribute)。

			\mintinline{C}{__attribute__}的书写特征是:\mintinline{C}{__attribute__}前后都有两个下划线,并切后面会紧跟一对原括弧,括弧里面是相应的\mintinline{C}{__attribute__}参数。

			\mintinline{C}{__attribute__}的语法格式为:\mintinline{C}{__attribute__}((attribute-list))

			\mintinline{C}{__attribute__}的位置约束为:放于声明的尾部“;”之前。
			参考博客:http://www.cnblogs.com/astwish/p/3460618.html
			关键字\mintinline{C}{__attribute__}也可以对结构体(struct)或共用体(union)进行属性设置。大致有六个参数值可以被设定,即:aligned, packed, \mintinline{C}{transparent_union}, unused, deprecated 和 \mintinline{C}{may_alias} 。

			在使用\mintinline{C}{__attribute__}参数时,你也可以在参数的前后都加上两个下划线,例如,使用\mintinline{C}{__aligned__}而不是aligned ,这样,你就可以在相应的头文件里使用它而不用关心头文件里是否有重名的宏定义。aligned (alignment)该属性设定一个指定大小的对齐格式(以字节 为单位)。		

			\subsubsection{设置变量属性}
				下面的声明将强制编译器确保(尽它所能)变量类型为\mintinline{C}{int32_t}的变量在分配空间时采用8字节对齐方式。

\begin{minted}[linenos]{C}
typedef int int32_t __attribute__ ((aligned(8)));
\end{minted}
			\subsubsection{设置类型属性}

				下面的声明将强制编译器确保(尽它所能)变量类型为struct S的变量在分配空间时采用8字节对齐方式。				
\begin{minted}[linenos]{C}
struct S {
	short b[3];
} __attribute__ ((aligned (8)));
\end{minted}

			如上所述,你可以手动指定对齐的格式,同样,你也可以使用默认的对齐方式。如果aligned后面不紧跟一个指定的数字值,那么编译器将依据你的目标机器情况使用最大最有益的对齐方式。例如:

\begin{minted}[linenos]{C}
struct S {
	short b[3];
} __attribute__ ((aligned));
\end{minted}

			这里,如果sizeof(short)的大小为2(byte),那么,S 的大小就为6 。取一个2 的次方值,使得该值大于等于6 ,则该值为8,所以编译器将设置S类型的对齐方式为8字节。aligned 属性使被设置的对象占用更多的空间,相反的,使用packed 可以减小对象占用的空间。
			需要注意的是,attribute 属性的效力与你的连接器也有关,如果你的连接器最大只支持16 字节对齐,那么你此时定义32 字节对齐也是无济于事的。

			\subsubsection{设置函数属性}
	\subsection{分支预测优化}

		现代处理器均为流水线结构。而分支语句可能导致流水线断流。因此,很多处理器均有分支预测的功能。
然而,分支预测失败所导致的惩罚也是相对高昂的。为了提升性能,Linux的很多分支判断中都使用了
\mintinline{c}{likely()}和\mintinline{c}{unlikely()}这组宏定义来人工指示编译器,哪些
分支出现的概率极高,以便编译器进行优化。

		这里有两种定义,一种是开启了分支语句分析相关的选项时,内核会采用下面的一种定义
\begin{minted}[linenos]{c}
/* include/linux/compiler.h
 * 采用__builtin_constant_p(x)来忽略常量表达式。
 */
# ifndef likely
#  define likely(x)     (__builtin_constant_p(x) ? !!(x) : __branch_check__(x, 1))
# endif
# ifndef unlikely
#  define unlikely(x)   (__builtin_constant_p(x) ? !!(x) : __branch_check__(x, 0))
# endif
\end{minted}

		在该定义下,\mintinline{c}{__branch_check__}用于跟踪分支结果并更新统计数据。

		如果不开启该选项,则定义得较为简单:
\begin{minted}[linenos]{c}
# define likely(x)      __builtin_expect(!!(x), 1)
# define unlikely(x)    __builtin_expect(!!(x), 0)
\end{minted}
其中\mintinline{c}{__builtin_expect}是GCC的内置函数,用于指示编译器,该条件语句最可能的结果是什么。

	\section{Sparse}

		\subsection{\mintinline{C}{__bitwise}}
\begin{minted}[linenos]{C}
#define __bitwise    __attribute__((bitwise))     
\end{minted}
		
		该宏主要是为了确保变量是相同的位方式(比如 bit-endian, little-endiandeng),对于使用了\mintinline{C}{__bitwise}宏的变量,Sparse会检查这个变量是否一直在同一种位方式(big-endian, little-endian或其他)下被使用。如果此变量再多个位方式下被使用了,Sparse会给出警告。例子:

\begin{minted}[linenos]{C}
#ifdef __CHECKER__
#define __bitwise__ __attribute__((bitwise))
#else
#define __bitwise__
#endif
#ifdef __CHECK_ENDIAN__
#define __bitwise __bitwise__
#else
#define __bitwise
#endif

typedef __u16 __bitwise __le16;
typedef __u16 __bitwise __be16;
typedef __u32 __bitwise __le32;
typedef __u32 __bitwise __be32;
typedef __u64 __bitwise __le64;
typedef __u64 __bitwise __be64;
\end{minted}
		
	\section{操作系统}
		\subsection{RCU}
			RCU(Read-Copy Update),是数据同步的一种方式,在当前的Linux内核中发挥着重要的作用。RCU主要针对的数据对象是链表,目的是提高遍历读取数据的效率,为了达到目的使用RCU机制读取数据的时候不对链表进行耗时的加锁操作。这样在同一时间可以有多个线程同时读取该链表,并且允许一个线程对链表进行修改(修改的时候,需要加锁)。RCU适用于需要频繁的读取数据,而相应修改数据并不多的情景,例如在文件系统中,经常需要查找定位目录,而对目录的修改相对来说并不多,这就是RCU发挥作用的最佳场景。

			Linux内核源码当中,关于RCU的文档比较齐全,你可以在 /Documentation/RCU/ 目录下找到这些文件。

			在RCU的实现过程中,我们主要解决以下问题:

       			1、在读取过程中,另外一个线程删除了一个节点。删除线程可以把这个节点从链表中移除,但它不能直接销毁这个节点,必须等到所有的读取线程读取完成以后,才进行销毁操作。RCU中把这个过程称为宽限期(Grace period)。

       			2、在读取过程中,另外一个线程插入了一个新节点,而读线程读到了这个节点,那么需要保证读到的这个节点是完整的。这里涉及到了发布-订阅机制(Publish-Subscribe Mechanism)。

       			3、保证读取链表的完整性。新增或者删除一个节点,不至于导致遍历一个链表从中间断开。但是RCU并不保证一定能读到新增的节点或者不读到要被删除的节点。
	\section{CPU}
		
	\section{存储系统}
		\subsection{字节序}
			在计算机中内存存储一般分为大端和小端两种。而在网络传输的过程中,大小端的不一致会带来问题。
因此,网络协议中对于字节序都有明确规定。一般采用大端序。

			Linux中,对于这一部分的支持放在了\mintinline{text}{include/linux/byteorder/generic.h}
中。而实现,则交由体系结构相关的代码来完成。

\begin{minted}[linenos]{c}
/* 下面的函数用于进行对16位整型或者32位整型在网络传输格式和本地格式之间的转换。
 */
ntohl(__u32 x)
ntohs(__u16 x)
htonl(__u32 x)
htons(__u16 x)
\end{minted}

			上面函数的命名规则是末尾的l代表32位,s代表16位。n代表network,h代表host。
根据命名规则,不难知道函数的用途。比如htons就是从本地的格式转换的网络传输用的格式,
转换的是16位整数。
