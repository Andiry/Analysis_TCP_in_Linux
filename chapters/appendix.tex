\chapter{附录:基础知识}
	\section{C语言}
		\subsection{结构体初始化}
\begin{minted}[linenos]{C}
typedef struct{
	int  a;
	char ch;
}flag;
/*
	目的是将a初始化为1,ch初始化为'u'.
*/
/*法一:分别初始化*/
flag tmp;
tmp.a=1;
tmp.ch='u';

/*法二:点式初始化*/
flag tmp={.a=1,.ch='u'};		//注意两个变量之间使用 , 而不是;
/*法三:*/
flag tmp={
			a:1,
			ch:'u'
		};
\end{minted}
			当然,我们也可以使用上述任何一种方法只对结构体中的某几个变量进行初始化。

		\subsection{位字段}
			在存储空间极为宝贵的情况下,有可能需要将多个对象保存在一个机器字中。而在linux开发的早期,那时确实空间极其宝贵。于是乎,那一帮黑客们就发明了各种各样的办法。一种常用的办法是使用类似于编译器符号表的单个二进制位标志集合,即定义一系列的2的指数次方的数,此方法确实有效。但是,仍然十分浪费空间。而且有可能很多位都利用不到。于是乎,他们提出了另一种新的思路即位字段。我们可以利用如下方式定义一个包含3位的变量。

\begin{minted}[linenos]{C}
struct {
	unsigned int a:1;
	unsigned int b:1;
	unsigned int c:1;
}flags;
\end{minted}
			
			字段可以不命名,无名字段,即只有一个冒号和宽度,起到填充作用。特殊宽度0可以用来强制在下一个字边界上对齐,一般位于结构体的尾部。
			
			冒号后面表示相应字段的宽度(二进制宽度),即不一定非得是1位。字段被声明为\mintinline{C}{unsigned int}类型,以确保它们是无符号量。

			当然我们需要注意,机器是分大端和小端存储的。因此,我们在选择外部定义数据的情况下爱,必须仔细考虑那一端优先的问题。同时,字段不是数组,并且没有地址,因此不能对它们使用\mintinline{C}{&}运算符。	
	\section{操作系统}
