\chapter{附录:基础知识}

\minitoc

	\section{C语言}
		\subsection{结构体初始化}
\begin{minted}[linenos]{C}
typedef struct{
	int  a;
	char ch;
}flag;
/*
	目的是将a初始化为1,ch初始化为'u'.
*/
/*法一:分别初始化*/
flag tmp;
tmp.a=1;
tmp.ch='u';

/*法二:点式初始化*/
flag tmp={.a=1,.ch='u'};		//注意两个变量之间使用 , 而不是;
/*法三:*/
flag tmp={
			a:1,
			ch:'u'
		};
\end{minted}
			当然,我们也可以使用上述任何一种方法只对结构体中的某几个变量进行初始化。

		\subsection{位字段}
			在存储空间极为宝贵的情况下,有可能需要将多个对象保存在一个机器字中。而在linux开发的早期,那时确实空间极其宝贵。于是乎,那一帮黑客们就发明了各种各样的办法。一种常用的办法是使用类似于编译器符号表的单个二进制位标志集合,即定义一系列的2的指数次方的数,此方法确实有效。但是,仍然十分浪费空间。而且有可能很多位都利用不到。于是乎,他们提出了另一种新的思路即位字段。我们可以利用如下方式定义一个包含3位的变量。

\begin{minted}[linenos]{C}
struct {
	unsigned int a:1;
	unsigned int b:1;
	unsigned int c:1;
}flags;
\end{minted}
			
			字段可以不命名,无名字段,即只有一个冒号和宽度,起到填充作用。特殊宽度0可以用来强制在下一个字边界上对齐,一般位于结构体的尾部。
			
			冒号后面表示相应字段的宽度(二进制宽度),即不一定非得是1位。字段被声明为\mintinline{C}{unsigned int}类型,以确保它们是无符号量。

			当然我们需要注意,机器是分大端和小端存储的。因此,我们在选择外部定义数据的情况下爱,必须仔细考虑那一端优先的问题。同时,字段不是数组,并且没有地址,因此不能对它们使用\mintinline{C}{&}运算符。	
	\section{GCC}
		\subsection{\mintinline{C}{__attribute__}}
			GNU C的一大特色就是\mintinline{C}{__attribute__}机制。\mintinline{C}{__attribute__}可以设置函数属性(Function Attribute)、变量属性(Variable Attribute)和类型属性(Type Attribute)。

			\mintinline{C}{__attribute__}的书写特征是:\mintinline{C}{__attribute__}前后都有两个下划线,并切后面会紧跟一对原括弧,括弧里面是相应的\mintinline{C}{__attribute__}参数。

			\mintinline{C}{__attribute__}的语法格式为:\mintinline{C}{__attribute__}((attribute-list))

			\mintinline{C}{__attribute__}的位置约束为:放于声明的尾部“;”之前。
			参考博客:http://www.cnblogs.com/astwish/p/3460618.html
			关键字\mintinline{C}{__attribute__}也可以对结构体(struct)或共用体(union)进行属性设置。大致有六个参数值可以被设定,即:aligned, packed, \mintinline{C}{transparent_union}, unused, deprecated 和 \mintinline{C}{may_alias} 。

			在使用\mintinline{C}{__attribute__}参数时,你也可以在参数的前后都加上两个下划线,例如,使用\mintinline{C}{__aligned__}而不是aligned ,这样,你就可以在相应的头文件里使用它而不用关心头文件里是否有重名的宏定义。aligned (alignment)该属性设定一个指定大小的对齐格式(以字节 为单位)。		

			\subsubsection{设置变量属性}
				下面的声明将强制编译器确保(尽它所能)变量类型为\mintinline{C}{int32_t}的变量在分配空间时采用8字节对齐方式。

\begin{minted}[linenos]{C}
typedef int int32_t __attribute__ ((aligned(8)));
\end{minted}
			\subsubsection{设置类型属性}

				下面的声明将强制编译器确保(尽它所能)变量类型为struct S的变量在分配空间时采用8字节对齐方式。				
\begin{minted}[linenos]{C}
struct S {
	short b[3];
} __attribute__ ((aligned (8)));
\end{minted}

			如上所述,你可以手动指定对齐的格式,同样,你也可以使用默认的对齐方式。如果aligned后面不紧跟一个指定的数字值,那么编译器将依据你的目标机器情况使用最大最有益的对齐方式。例如:

\begin{minted}[linenos]{C}
struct S {
	short b[3];
} __attribute__ ((aligned));
\end{minted}

			这里,如果sizeof(short)的大小为2(byte),那么,S 的大小就为6 。取一个2 的次方值,使得该值大于等于6 ,则该值为8,所以编译器将设置S类型的对齐方式为8字节。aligned 属性使被设置的对象占用更多的空间,相反的,使用packed 可以减小对象占用的空间。
			需要注意的是,attribute 属性的效力与你的连接器也有关,如果你的连接器最大只支持16 字节对齐,那么你此时定义32 字节对齐也是无济于事的。

			\subsubsection{设置函数属性}
	\subsection{分支预测优化}

		现代处理器均为流水线结构。而分支语句可能导致流水线断流。因此,很多处理器均有分支预测的功能。
然而,分支预测失败所导致的惩罚也是相对高昂的。为了提升性能,Linux的很多分支判断中都使用了
\mintinline{c}{likely()}和\mintinline{c}{unlikely()}这组宏定义来人工指示编译器,哪些
分支出现的概率极高,以便编译器进行优化。

		这里有两种定义,一种是开启了分支语句分析相关的选项时,内核会采用下面的一种定义
\begin{minted}[linenos]{c}
/* include/linux/compiler.h
 * 采用__builtin_constant_p(x)来忽略常量表达式。
 */
# ifndef likely
#  define likely(x)     (__builtin_constant_p(x) ? !!(x) : __branch_check__(x, 1))
# endif
# ifndef unlikely
#  define unlikely(x)   (__builtin_constant_p(x) ? !!(x) : __branch_check__(x, 0))
# endif
\end{minted}

		在该定义下,\mintinline{c}{__branch_check__}用于跟踪分支结果并更新统计数据。

		如果不开启该选项,则定义得较为简单:
\begin{minted}[linenos]{c}
# define likely(x)      __builtin_expect(!!(x), 1)
# define unlikely(x)    __builtin_expect(!!(x), 0)
\end{minted}
其中\mintinline{c}{__builtin_expect}是GCC的内置函数,用于指示编译器,该条件语句最可能的结果是什么。

	\section{操作系统}
		\subsection{字节序}
			CPU分为大端和小端两种。而在网络传输的过程中,大小端的不一致会带来问题。
因此,网络协议中对于字节序都有明确规定。一般采用大端序。

			Linux中,对于这一部分的支持放在了\mintinline{text}{include/linux/byteorder/generic.h}
中。而实现,则交由体系结构相关的代码来完成。

\begin{minted}[linenos]{c}
/* 下面的函数用于进行对16位整型或者32位整型在网络传输格式和本地格式之间的转换。
 */
ntohl(__u32 x)
ntohs(__u16 x)
htonl(__u32 x)
htons(__u16 x)
\end{minted}

			上面函数的命名规则是末尾的l代表32位,s代表16位。n代表network,h代表host。
根据命名规则,不难知道函数的用途。比如htons就是从本地的格式转换的网络传输用的格式,
转换的是16位整数。
